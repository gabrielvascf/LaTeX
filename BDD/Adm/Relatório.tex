\documentclass[a4paper, 12pt]{article}
% --- PACOTES ---
\usepackage[utf8]{inputenc}
\usepackage{listings}
\usepackage[T1]{fontenc}
\usepackage[brazil]{babel}
\usepackage{geometry}
\geometry{a4paper, margin=2.5cm}
\usepackage{listings}
\usepackage{xcolor}
\usepackage{hyperref}
\hypersetup{
    colorlinks=true,
    linkcolor=blue,
    filecolor=magenta,      
    urlcolor=cyan,
    pdftitle={Relatório de Laboratório: Administração de Usuários Oracle},
}
\usepackage{amsmath}
\usepackage{graphicx}
\graphicspath{Figuras}

% --- CONFIGURAÇÃO DO AMBIENTE DE CÓDIGO (LISTINGS) ---
\definecolor{codegreen}{rgb}{0,0.6,0}
\definecolor{codegray}{rgb}{0.5,0.5,0.5}
\definecolor{codepurple}{rgb}{0.58,0,0.82}
\definecolor{backcolour}{rgb}{0.95,0.95,0.92}

\lstdefinestyle{sqlstyle}{
    backgroundcolor=\color{backcolour},   
    commentstyle=\color{codegreen},
    keywordstyle=\color{magenta},
    numberstyle=\tiny\color{codegray},
    stringstyle=\color{codepurple},
    basicstyle=\footnotesize\ttfamily,
    breakatwhitespace=false,         
    breaklines=true,                 
    captionpos=b,                    
    keepspaces=true,                 
    numbers=left,                    
    numbersep=5pt,                  
    showspaces=false,                
    showstringspaces=false,
    showtabs=false,                  
    tabsize=2,
    language=SQL,
    morekeywords={*,CONNECT, RESOURCE, IDENTIFIED, tablespace, quota, unlimited, grantee, VARCHAR2, DBA, ROLE, ANY, TABLE, VIEW}
}
\lstset{style=sqlstyle}

% --- INFORMAÇÕES DO DOCUMENTO ---
\title{Relatório de Laboratório: Administração de Usuários Oracle}
\author{Gabriel Vasconcelos Ferreira}
\date{\today}

% --- INÍCIO DO DOCUMENTO ---
\begin{document}

\maketitle
\newpage
\section{Conexão com SYSTEM e Análise Inicial}
A primeira etapa consiste em conectar-se com o usuário \texttt{SYSTEM}, que possui privilégios administrativos, e verificar informações básicas sobre o ambiente.

\subsection{Visão \texttt{v\$version}}
\begin{lstlisting}
SELECT * FROM v$version;
\end{lstlisting}
\textbf{Finalidade:} A visão dinâmica \texttt{v\$version} fornece informações sobre a versão do software Oracle Database e seus componentes instalados. Ela é útil para identificar a edição do banco de dados (ex: Enterprise, Standard, Express), a versão principal e de patch, e a compatibilidade do hardware. O resultado mostra, por exemplo:
\begin{itemize}
	\item Oracle Database 19c Enterprise Edition Release 19.0.0.0.0
	\item PL/SQL Release 19.0.0.0.0
	\item TNS for Linux: Version 19.0.0.0.0
\end{itemize}

\subsection{Visão \texttt{dba\_users}}
\begin{lstlisting}
SELECT username FROM dba_users;
\end{lstlisting}
\textbf{Finalidade:} A visão \texttt{dba\_users} é parte do dicionário de dados e contém informações sobre todos os usuários definidos no banco de dados. Um administrador (DBA) pode usar esta visão para listar os usuários existentes, verificar seus status (ex: \texttt{OPEN}, \texttt{LOCKED}), \textit{tablespaces} padrão, data de criação, etc. A consulta retorna uma lista com todos os nomes de usuário cadastrados na instância.

\section{Criação e Configuração do Usuário USR\_LAB01}

\subsection{Comando \texttt{CREATE USER}}
\begin{lstlisting}
CREATE USER USR_LAB01 
  IDENTIFIED BY SENHA 
  DEFAULT TABLESPACE users 
  QUOTA UNLIMITED ON users;
\end{lstlisting}
\textbf{Explicação do Comando:}
\begin{itemize}
	\item \texttt{CREATE USER USR\_LAB01}: Comando que inicia a criação de um novo usuário chamado \texttt{USR\_LAB01}.
	\item \texttt{IDENTIFIED BY SENHA}: Define a senha inicial do usuário como \texttt{SENHA}.
	\item \texttt{DEFAULT TABLESPACE users}: Especifica que todos os objetos criados por este usuário (como tabelas e índices) serão armazenados na \textit{tablespace} chamada \texttt{users}, a menos que outra seja explicitamente indicada.
	\item \texttt{QUOTA UNLIMITED ON users}: Concede ao usuário uma cota ilimitada de espaço de armazenamento na \textit{tablespace} \texttt{users}. Sem uma cota, o usuário não poderia criar objetos nela.
\end{itemize}

\subsection{Comando \texttt{GRANT CONNECT, RESOURCE}}
\begin{lstlisting}
GRANT CONNECT, RESOURCE to USR_LAB01;
\end{lstlisting}
\textbf{Finalidade das Roles \texttt{CONNECT} e \texttt{RESOURCE}:}
\begin{itemize}
	\item \textbf{\texttt{CONNECT}}: É uma \textit{role} que agrupa privilégios básicos de sistema. O mais fundamental é o \texttt{CREATE SESSION}, que permite ao usuário conectar-se ao banco de dados. Sem este privilégio, a tentativa de login falharia com o erro "ORA-01045: user lacks CREATE SESSION privilege".
	\item \textbf{\texttt{RESOURCE}}: Esta \textit{role} concede privilégios para criar objetos no próprio \textit{schema} do usuário, como \texttt{CREATE TABLE}, \texttt{CREATE PROCEDURE}, \texttt{CREATE TRIGGER}, entre outros.
\end{itemize}
\textbf{Observação:} A Oracle atualmente recomenda a criação de \textit{roles} customizadas com os privilégios mínimos necessários, em vez de usar \texttt{CONNECT} e \texttt{RESOURCE}, que podem conceder mais permissões do que o necessário.

\subsection{Teste de Conexão}
Após a criação e a concessão da \textit{role} \texttt{CONNECT}, abriu-se uma nova janela de terminal/SQL Developer e foi possível conectar com o usuário \texttt{USR\_LAB01} e a senha \texttt{SENHA}.
\begin{center}
	\includegraphics[scale=0.5]{Figuras/LoginUSR_LAB01.png}
\end{center}

\section{Gerenciamento de Senha e Sessão}

\subsection{Alteração de Senha}
O comando abaixo foi executado na janela do usuário \texttt{SYSTEM}.
\begin{lstlisting}
ALTER USER USR_LAB01 IDENTIFIED BY new_password;
\end{lstlisting}
Este comando altera a senha do usuário \texttt{USR\_LAB01} para \texttt{new\_password}.

\subsection{Verificação da Sessão Ativa}
Na janela onde o usuário \texttt{USR\_LAB01} já estava conectado, o seguinte comando foi executado:
\begin{lstlisting}
SELECT table_name FROM all_tables;
\end{lstlisting}
\textbf{Análise:} O comando funcionou normalmente. A alteração de senha por um DBA \textbf{não invalida as sessões já existentes} daquele usuário. A nova senha só será exigida para novas tentativas de conexão.

\subsection{Tentativa de Nova Conexão}
Após encerrar a sessão de \texttt{USR\_LAB01}, tentou-se conectar novamente.
\begin{itemize}
	\item \textbf{Usando a senha antiga (\texttt{SENHA}):} A conexão falhou com o erro "ORA-01017: invalid username/password".
	\item \textbf{Usando a nova senha (\texttt{new\_password}):} A conexão foi bem-sucedida.
\end{itemize}
\textbf{O que aconteceu?} O comando \texttt{ALTER USER} efetivou a troca da senha no dicionário de dados. Qualquer nova autenticação deve, obrigatoriamente, usar a credencial atualizada.

\section{Propriedade e Criação de Objetos}
Esta seção explora como a propriedade de objetos (\textit{schema}) funciona no Oracle.

\subsection{Criação de Tabelas pelo \texttt{SYSTEM}}
Os comandos foram executados na janela do usuário \texttt{SYSTEM}.
\begin{lstlisting}
SHOW USER; -- Retorna: USER is "SYSTEM"
CREATE TABLE xyz (name VARCHAR2(30));
\end{lstlisting}
\textbf{Em qual usuário a tabela foi criada?} A tabela \texttt{xyz} foi criada no \textit{schema} do usuário que executou o comando, ou seja, no \textit{schema} \texttt{SYSTEM}.

\begin{lstlisting}
CREATE TABLE USR_LAB01.xyz (name VARCHAR2(30));
\end{lstlisting}
\textbf{Em qual usuário a tabela foi criada?} A tabela \texttt{xyz} foi criada no \textit{schema} do usuário \texttt{USR\_LAB01}.\\
\textbf{Que nível de privilégio foi necessário?} Para criar um objeto no \textit{schema} de outro usuário, é necessário o privilégio de sistema \texttt{CREATE ANY TABLE}. O usuário \texttt{SYSTEM} possui este privilégio por padrão.

\subsection{Verificação de Propriedade pelo \texttt{USR\_LAB01}}
Na janela do \texttt{USR\_LAB01}, os comandos abaixo foram executados.
\begin{lstlisting}
DESC xyz
\end{lstlisting}
O comando funcionou, exibindo a estrutura da tabela \texttt{xyz}, confirmando que ela pertence ao \textit{schema} \texttt{USR\_LAB01}.

\begin{lstlisting}
DESC system.xyz
\end{lstlisting}
\textbf{Esse comando funcionou?} Não. O comando resultou no erro "ORA-04043: object system.xyz does not exist".\\
\includegraphics[width=\textwidth]{Figuras/SHOW_SYSXYZ.png}
\textbf{O que falta ao usuário \texttt{USR\_LAB01}?} Falta o privilégio de acessar objetos do \textit{schema} \texttt{SYSTEM}. Para que o comando \texttt{DESC} (e outros como \texttt{SELECT}) funcione, o usuário \texttt{USR\_LAB01} precisaria receber o privilégio \texttt{SELECT} sobre a tabela \texttt{system.xyz} ou um privilégio mais amplo como \texttt{SELECT ANY TABLE}.

\section{Concessão de Privilégios de Objeto}
Nesta etapa, criamos um segundo usuário e concedemos a ele acesso a um objeto específico do \texttt{USR\_LAB01}.

\subsection{Criação do \texttt{USR\_LAB02} e Concessão de Privilégios}
Na janela do \texttt{SYSTEM}:
\begin{lstlisting}
CREATE USER USR_LAB02 IDENTIFIED BY SENHA DEFAULT TABLESPACE users;

GRANT INSERT, DELETE, SELECT ON USR_LAB01.XYZ TO USR_LAB02;

GRANT CONNECT TO USR_lab02;
\end{lstlisting}
\textbf{Que operação está acontecendo?} O comando \texttt{GRANT INSERT, DELETE, SELECT ON USR\_LAB01.XYZ TO USR\_LAB02;} está concedendo privilégios de objeto. Especificamente, o usuário \texttt{USR\_LAB02} está recebendo permissão para executar as operações de \texttt{INSERT}, \texttt{DELETE} e \texttt{SELECT} na tabela \texttt{XYZ} que pertence ao \textit{schema} \texttt{USR\_LAB01}.

\subsection{Verificação de Privilégios Concedidos}
\begin{lstlisting}
select * from dba_tab_privs where grantee = 'USR_LAB02';
\end{lstlisting}
\textbf{Qual o significado do resultado?} Esta consulta à visão \texttt{dba\_tab\_privs} (privilégios de tabelas) mostra todos os privilégios de objeto concedidos a \texttt{USR\_LAB02}. O resultado exibirá três linhas, indicando que o \texttt{GRANTEE} (beneficiário) \texttt{USR\_LAB02} recebeu os privilégios (\texttt{PRIVILEGE}) \texttt{INSERT}, \texttt{DELETE} e \texttt{SELECT} na tabela (\texttt{TABLE\_NAME}) \texttt{XYZ} do proprietário (\texttt{OWNER}) \texttt{USR\_LAB01}.\\
\includegraphics[width=\textwidth]{Figuras/SELECT_DBA_TAB_PRIVS.png}

\subsection{Operações do \texttt{USR\_LAB02}}
Conectado como \texttt{USR\_LAB02}:
\begin{lstlisting}
-- Insercao de dados na tabela de outro schema 
insert into usr_lab01.xyz values ('teste de nome');
commit;

-- Consulta aos dados
select * from usr_lab01.xyz;
\end{lstlisting}
\textbf{Resultado e Explicação:} O comando \texttt{SELECT} funcionou e retornou a linha inserida: `'teste de nome'`. Ele funcionou porque o usuário \texttt{USR\_LAB02} recebeu explicitamente o privilégio \texttt{SELECT} sobre a tabela \texttt{usr\_lab01.xyz}. O \texttt{INSERT} também funcionou pelo mesmo motivo.

\begin{lstlisting}
select * from system.xyz;
select * from xyz;
\end{lstlisting}
\textbf{Explicação do porquê NÃO funcionaram:}
\begin{itemize}
	\item \texttt{select * from system.xyz;}: Falhou porque \texttt{USR\_LAB02} não tem permissão de acesso à tabela \texttt{xyz} do \textit{schema} \texttt{SYSTEM}.
	\item \texttt{select * from xyz;}: Falhou porque, ao omitir o \textit{schema}, o Oracle procura por uma tabela chamada \texttt{xyz} no \textit{schema} do próprio usuário (\texttt{USR\_LAB02}). Como essa tabela não existe, um erro é retornado.
\end{itemize}

\section{Gerenciamento de Privilégios com Roles}
Esta seção demonstra como usar \textit{roles} para gerenciar o acesso a visões do dicionário de dados.

\subsection{Acesso Restrito ao Dicionário de Dados}
Na janela do \texttt{USR\_LAB01}:
\begin{lstlisting}
select * from dba_sys_privs;
\end{lstlisting}
\textbf{O comando funcionou?} Não. O acesso a visões \texttt{DBA\_} é restrito a usuários com altos privilégios, como \texttt{SYSTEM} ou aqueles com o privilégio \texttt{SELECT ANY DICTIONARY} ou a \textit{role} \texttt{SELECT\_CATALOG\_ROLE}.

\subsection{Criação e Concessão da Role \texttt{new\_dba}}
Na janela do \texttt{SYSTEM}:
\begin{lstlisting}
CREATE ROLE new_dba;
GRANT CONNECT TO new_dba;
GRANT SELECT ANY TABLE TO new_dba;
GRANT SELECT_CATALOG_ROLE TO new_dba;

GRANT new_dba TO USR_LAB01;
\end{lstlisting}

\subsection{Verificação do Acesso}
Na janela do \texttt{USR\_LAB01}, após reconectar para que os novos privilégios da \textit{role} sejam carregados na sessão:
\begin{lstlisting}
select * from dba_sys_privs;
\end{lstlisting}
\textbf{Explicação do processo:}
\begin{enumerate}
	\item Foi criada uma \textit{role} customizada chamada \texttt{new\_dba}.
	\item A esta \textit{role} foram concedidos privilégios de sistema importantes: \texttt{CONNECT}, \texttt{SELECT ANY TABLE} (permite ler dados de qualquer tabela em qualquer \textit{schema}, exceto \texttt{SYS}) e \texttt{SELECT\_CATALOG\_ROLE} (uma \textit{role} pré-definida que concede privilégio de leitura sobre as visões do dicionário de dados).
	\item A \textit{role} \texttt{new\_dba}, contendo todos esses privilégios, foi então concedida ao usuário \texttt{USR\_LAB01}.
	\item Ao herdar os privilégios da \textit{role}, \texttt{USR\_LAB01} passou a ter permissão para consultar a visão \texttt{dba\_sys\_privs} e muitas outras, tornando o comando bem-sucedido.
\end{enumerate}

\section{Análise Final de Privilégios}
As seguintes visões permitem auditar como os privilégios e \textit{roles} foram distribuídos.

\begin{itemize}
	\item \texttt{select * from dba\_sys\_privs}: Mostra os privilégios de sistema concedidos diretamente a usuários ou \textit{roles}.
	\item \texttt{SELECT * FROM DBA\_ROLE\_PRIVS}: Lista quais \textit{roles} foram concedidas a quais usuários (ex: mostra que \texttt{USR\_LAB01} possui a \textit{role} \texttt{NEW\_DBA}).
	\item \texttt{SELECT * FROM ROLE\_ROLE\_PRIVS}: Mostra \textit{roles} que foram concedidas a outras \textit{roles}.
	\item \texttt{SELECT * FROM ROLE\_SYS\_PRIVS}: Detalha quais privilégios de sistema uma \textit{role} possui (ex: mostra que \texttt{NEW\_DBA} tem o privilégio \texttt{SELECT ANY TABLE}).
	\item \texttt{SELECT * FROM ROLE\_TAB\_PRIVS}: Detalha quais privilégios de objeto (em tabelas, views, etc.) uma \textit{role} possui.
\end{itemize}

\end{document}
