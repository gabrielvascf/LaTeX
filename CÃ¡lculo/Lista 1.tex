\documentclass{jhwhw}
\usepackage{pgfplots}
\pgfplotsset{compat=1.18}
\usepackage{tikz}
\usepackage[portuguese]{babel}
\title{Cálculo}
\author{Gabriel Vasconcelos Ferreira}
\begin{document}
\maketitle
\chapter{Função constante / Função de 1º grau}
\problem{Faça o gráfico das funções:}
\begin{enumerate}
	\item $y = \pi$
	\item $y = -\frac{3}{2}$
	\item $y = -x + 5$
	\item $y = 2x + 4$
\end{enumerate}
\problem{A tarifa de táxi comum em São Paulo, em outubro de 2023, foi definida da seguinte forma: R\$ 6,00 de bandeirada (custo fixo) mais R\$ 4,25 por km rodado (custo variável). Qual é a fórmula ou regra que descreve essa situação? Apresente o gráfico dessa situação. Determine o valor a ser pago (custo total) por uma corrida relativa a um percurso de 5 km.}
\newpage
\soln{Problema 1: }
\soln{A) \(y = \pi\)}
\begin{center}
	\begin{tikzpicture}
		\begin{axis}[
				xlabel={$x$},
				ylabel={$y$},
				axis lines=middle,
			]
			\addplot[red,thick]{pi};
		\end{axis}
	\end{tikzpicture}
\end{center}
\soln{B) \(y = -\frac{3}{2}\)}
\begin{center}
	\begin{tikzpicture}
		\begin{axis}[
				xlabel={$x$},
				ylabel={$y$},
				xmin=-2, xmax = 2,
				ymin=-2, ymax = 2,
				domain=-2:2,
				samples=100,
				axis lines=middle,
			]
			\addplot[red,thick]{-3/2};
		\end{axis}
	\end{tikzpicture}
\end{center}
\soln{C) \(y =-x+5\)}
\begin{center}
	\begin{tikzpicture}
		\begin{axis}[
				xlabel={$x$},
				ylabel={$y$},
				axis lines=middle,
			]
			\addplot[red,thick]{-x+5};
		\end{axis}
	\end{tikzpicture}
\end{center}
\soln{D) \(y = 2x+4\)}
\begin{center}
	\begin{tikzpicture}
		\begin{axis}[
				xlabel={$x$},
				ylabel={$y$},
				axis lines=middle,
			]
			\addplot[red,thick]{2*x+4};
		\end{axis}
	\end{tikzpicture}
\end{center}
\newpage
\soln{Problema 2: }
\solution{
	\part{Fórmula:}
	\[f(t) = 6 + 4.25t\]
	\part{Gráfico da função:}
	\begin{center}
		\begin{tikzpicture}
			\begin{axis}[
					xlabel={$x$},
					ylabel={$y$},
					domain=0:6,
					axis lines=middle,
				]
				\addplot[red,thick]{6+4.25*x};
			\end{axis}
		\end{tikzpicture}
	\end{center}
	\part{Valor para 5km:}
	\begin{center}
		\(f(t) = 6 + 4.25t\)\\
		\(f(5) = 6 + 4.25*5\)\\
		\(f(5) = 6 + 21.25\)\\
		\(f(5) = 27.25\)\\
	\end{center}
}

\chapter{Função de 2º grau}
\setcounter{ProblemNum}{0}
\setcounter{SubProblemNum}{0}
\problem{Faça o gráfico das funções.}
\begin{enumerate}
	\item $y = -x^2 + 1$
	\item $y = x^2 - 2x$
	\item $y = x^2 -2x - 3$
	\item $y = -x^2 + 3x$
\end{enumerate}
\problem{
Um foguete e atirado para cima de modo que sua altura $h$, em relação ao solo, é dada, em função do tempo, pela função $h = 10 + 120t - 5t^2$, em que o tempo é dado em segundos e a altura é dada em metros. Calcule:
}
\begin{enumerate}
	\item A altura do foguete 2 segundos depois de ser lançado.
	\item O tempo necessário para o foguete atingir a altura de 485 metros.
\end{enumerate}
\problem{
A receita $R$ de uma pequena empresa, entre os dias 1 e 30 do mês, é dada, em função do dia $d$ do mês, pela função $R(d) = -d^2 + 31d - 30$, enquanto o custo $C$ é dada por $C(d) = 11d - 19$.
}
\begin{enumerate}
	\item Encontre a função lucro $L$, sendo que o lucro é igual à Receita menos Custo, ou seja, $L(d) = R(d) - C(d)$.
	\item Em que dias o lucro da empresa é zero?
\end{enumerate}
\newpage
\problem{
O saldo de uma conta bancária é dado por $S = t^2 - 11t + 24$, onde $S$ é o saldo em reais e $t$ é o tempo em dias. Determine:
}
\begin{enumerate}
	\item em que dias o saldo é zero;
	\item em que período o saldo é negativo;
	\item em que dia o saldo é mínimo;
	\item o saldo mínimo, em reais.
\end{enumerate}
\soln{Problema 1}
\part{
	$y = -x^2 + 1$
	\begin{center}
		\begin{tikzpicture}
			\begin{axis}[
					xlabel={$x$},
					ylabel={$y$},
					domain=-6:6,
					axis lines=middle,
					xmin=-6,
					ymin=-8,
					xmax =6,
					ymax= 2,
					grid = both,
					minor tick num = 1,
					enlargelimits={abs=0.5},
				]
				\addplot [red, thick, samples=50] {-x^2 + 1};
			\end{axis}
		\end{tikzpicture}
	\end{center}
}
\part{
	$y = x^2 - 2x$
	\begin{center}
		\begin{tikzpicture}
			\begin{axis}[
					xlabel={$x$},
					ylabel={$y$},
					axis lines=middle,
					xmin=-2,
					ymin=-1,
					xmax =4,
					ymax= 8,
					grid=both,
					minor tick num = 1,
					enlargelimits={abs=0.5},
				]
				\addplot [red, thick, samples=50] {x^2 - 2*x};
			\end{axis}
		\end{tikzpicture}
	\end{center}
}
\part{
	$y = x^2 -2x - 3$
	\begin{center}
		\begin{tikzpicture}
			\begin{axis}[
					xlabel={$x$},
					ylabel={$y$},
					axis lines=middle,
					xmin=-4,
					ymin=-4,
					xmax =8,
					ymax= 8,
					grid=both,
					minor tick num = 1,
					enlargelimits={abs=0.5},
				]
				\addplot [red, thick, samples=50] {x^2 - 2*x - 3};
			\end{axis}
		\end{tikzpicture}
	\end{center}
}
\part{
	$y = -x^2 + 3x$
	\begin{center}
		\begin{tikzpicture}
			\begin{axis}[
					xlabel={$x$},
					ylabel={$y$},
					axis lines=middle,
					xmin=-4,
					ymin=-4,
					xmax =4,
					ymax= 3,
					grid=both,
					minor tick num = 1,
					enlargelimits={abs=0.5},
				]
				\addplot [red, thick, samples=50] {-x^2 + 3*x};
			\end{axis}
		\end{tikzpicture}
	\end{center}
}
\soln{Problema 2: }
\setcounter{SubProblemNum}{0}
\(h = 10 + 120t - 5t^2\)
\part{
	A altura do foguete 2 segundos após ser lançado:
	\begin{gather*}
		h = 10 + 120t - 5t^2    \\
		h = 10 + 120*2 - 5(2^2) \\
		h = 10 + 240 - 5*4      \\
		h = 250 - 20            \\
		h = 230	\\
	\end{gather*}
}
\part{
	O tempo necessário para o foguete atingir a altura de 485 metros:
	\begin{gather*}
		h = 10 + 120t - 5t^2    \\
		485 = 10 + 120t - 5t^2 \\
		485 - 10 = 120t - 5t^2 \\
		\frac{475}{120} = -5t^2 \\
		\frac{\frac{485}{120}}{-5} = t^2
	\end{gather*}
}
\chapter{Função exponencial e logaritmica}
\setcounter{ProblemNum}{0}
\setcounter{SubProblemNum}{0}
\problem{Faça o gráfico das funções}
\begin{enumerate}
	\item $y = 3^x$
	\item $y = \frac{1}{3}^x$
	\item $y = log_3(x)$
	\item $y = log_e(x) = ln(x)$
\end{enumerate}
\problem{
A função $P(t) = 300000*2^{0.05t}$ fornece o número $P$ de milhares de habitantes de uma cidade, em função do tempo $t$, em anos, a partir do ano de 1990.
}
\begin{enumerate}
	\item Determine o número de habitantes dessa cidade tem $t = 0$, que corresponde ao ano de 1990.
	\item Quantos habitantes, aproximadamente, espera-se que ela tenha após 10 anos, ou seja, no ano 2000?
	\item Faça o gráfico da função.
\end{enumerate}
\newpage
\problem{
Ao observar, em um microscópio, uma cultura de bactérias, um cientista percebeu que elas se reproduzem como uma função exponencial. A lei de formação que relaciona a quantidade de bactérias existentes com o tempo é igual a $f(t) = Q * 2^{t-1}$, em que $Q$ é a quantidade inicial de bactérias e $t$ é o tempo em horas. Se nessa cultura havia, inicialmente, 700 bactérias, a quantidade de bactérias após 4 horas será de (apresente os cálculos):
}
\begin{enumerate}
	\item 7000
	\item 8700
	\item 15300
	\item 11200
	\item 5600
\end{enumerate}

\problem{
Vamos supor um automóvel, com valor inicial de R\$109.000,00. Para fins contábeis, a Receita Federal estipula que a taxa de depreciação de veículos é de 20\% ao ano. Essa taxa só é utilizada para fins contábeis.
}
\begin{enumerate}
	\item Apresente a função dessa situação através de uma fórmula.
	\item Qual seria o valor aproximado do automóvel, 5 anos após o momento inicial?
	\item Qual o valor aproximado do automóvel após 10, 15 e 20 de compra?
	\item Faça o gráfico da função.
\end{enumerate}

\chapter{Função seno e cosseno}
\setcounter{ProblemNum}{0}
\setcounter{SubProblemNum}{0}
\problem{Faça o gráfico das funções:}
\begin{enumerate}
	\item $y = 2 cos(x)$
	\item $y = 1 + cos(x)$
	\item $y = 2 + sin(x)$
	\item $y = sin(\frac{x}{2})$
\end{enumerate}

\end{document}
