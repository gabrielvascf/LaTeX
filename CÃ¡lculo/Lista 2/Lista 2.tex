\documentclass{jhwhw}
\usepackage{pgfplots}
\pgfplotsset{compat=1.18}
\usepackage{mathtools}
\usepackage{tikz}
\usepackage[portuguese]{babel}
\title{Cálculo\\Lista 2 - Limites}
\author{Gabriel Vasconcelos Ferreira}
\begin{document}
\maketitle
\chapter{Noção Intuitiva / Limites laterais}
\problem{
Dadas as funções e seus respectivos gráficos:
}
\begin{enumerate}
    \item calcule os limites laterais da $f(x)$
    \item compare os limites laterais e verifique se eles são iguais ou diferentes
    \item conclua se existe o limite da função e se existir, indique qual é o seu valor
\end{enumerate}
\begin{enumerate}
    \item [1)]$$
              \lim _{x \to 0} f(x), \operatorname{para} f(x)= \begin{cases}
                  x^2 & \text {, se } x \leq 0 \\ 1+x^2 & \text {, se } x>0
              \end{cases}$$
    \item [2)]$$
              \lim _{x \to 2} f(x), \operatorname{para} f(x)= \begin{cases}
                  x^2 & \text {, se } x \geq 2 \\ x+1 & \text {, se } x<2
              \end{cases}
          $$
    \item [3)]$$
              \lim _{x \to 0} f(x), \operatorname{para} f(x)= \begin{cases}
                  -x^2 + 3 & \text {, se } x < 0 \\ -4x + 3 & \text {, se } x\geq0
              \end{cases}
          $$
    \item [4)]$$
              \lim _{x \to 1} f(x), \operatorname{para} f(x)= \begin{cases}
                  3x+1 & \text {, se } x \neq 1 \\ 0 & \text {, se } x=1
              \end{cases}
          $$
\end{enumerate}
\newpage
\soln{1)$$
        \lim _{x \to 0} f(x), \operatorname{para} f(x)= \begin{cases}
            x^2 & \text {, se } x \leq 0 \\ 1+x^2 & \text {, se } x>0
        \end{cases}$$}
\begin{align*}
    \lim _{x \rightarrow 0^-} & f(x) = x^2 & \lim_{x\to0^+} & f(x) = 1+x^2   \\
                              & f(0) = 0^2 &                & f(0) = 1 + 0^2 \\
                              & f(0) = 0   &                & f(0) = 1       \\
    \lim _{x \rightarrow 0^-} & f(x) = 0   & \lim_{x\to0^+} & f(x) = 1
\end{align*}
\[
    \lim _{x \rightarrow 0^-} f(x) \neq \lim _{x \rightarrow 0^+} f(x)
\]
\soln{2)$$
        \lim _{x \to 2} f(x), \operatorname{para} f(x)= \begin{cases}
            x^2 & \text {, se } x \geq 2 \\ x+1 & \text {, se } x<2
        \end{cases}
    $$}
\begin{align*}
    \lim_{x \to 2^-} & f(x) = x^2 & \lim_{x\to2^+} & f(x) = x + 1 \\
                     & f(2) = 2^2 &                & f(2) = 2 + 1 \\
                     & f(2) = 4   &                & f(2) = 3     \\
    \lim_{x \to 2^-} & f(x) = 4   & \lim_{x\to2^+} & f(x) = 3
\end{align*}
\[
    \lim_{x \to 2^-} f(x) \neq \lim_{x\to2^+} f(x)
\]
\soln{3)$$
        \lim _{x \to 0} f(x), \operatorname{para} f(x)= \begin{cases}
            -x^2 + 3 & \text {, se } x < 0 \\ -4x + 3 & \text {, se } x \geq 0
        \end{cases}
    $$}
\begin{align*}
    \lim_{x \to 1^-} f(x) & = \lim_{x \to 1^-} (-x^2+3) & \lim_{x \to 1^+} f(x) & = \lim_{x \to 1^+} (-4x+3) \\
                          & = -1^2+3                    &                       & = -4*1+3                   \\
                          & = -1 + 3                    &                       & = -4+3                     \\
    \lim_{x \to 1^-} f(x) & = 2                         & \lim_{x \to 1^+} f(x) & = -1
\end{align*}
\[
    \lim_{x \to 1^-} f(x) \neq \lim_{x \to 1^+} f(x)
\]
\soln{4)$$
        \lim _{x \to 1} f(x), \operatorname{para} f(x)= \begin{cases}
            3x+1 & \text {, se } x \neq 1 \\ 0 & \text {, se } x=1
        \end{cases}
    $$}
\begin{align*}
    \lim_{x \to 1} f(x) & = \lim_{x \to 1} (3x+1) \\
                        & = 3*1 + 1               \\
    \lim_{x \to 1} f(x) & = 4
\end{align*}
\chapter{Limites com indeterminação}
\problem{Calcule os limites, indicando o passo a passo.}
\begin{enumerate}
    \item [1)] \[
              \lim_{x \to 0} \frac{\sqrt{x+3}-\sqrt{3}}{x}
          \]
    \item [2)] \[
              \lim_{x \to 10} \frac{x^2-100}{x-10}
          \]
    \item [3)] \[
              \lim_{x \to 7} \frac{x^2-49}{x-7}
          \]
    \item [4)] \[
              \lim_{x \to 2} \frac{3x^2-5x-2}{x^2+3x-10}
          \]
    \item [5)] \[
              \lim_{x \to -1} \frac{x^2+3x+2}{x^2-1}
          \]
\end{enumerate}
\newpage
\soln{1)}
$$\lim_{x \to 0} \frac{\sqrt{x+3}-\sqrt{3}}{x}$$
Multiplicando divisor e numerador pelo conjugado do numerador:
\begin{multline*}
    \frac{\sqrt{x+3}-\sqrt{3}}{x} =\\ \frac{(\sqrt{x+3}-\sqrt{3})(\sqrt{x+3}+\sqrt{3})}{x(\sqrt{x+3}+\sqrt{3})} = 
    \frac{x}{x(\sqrt{x+3}+\sqrt{3})} = 
    \frac{1}{\sqrt{x+3}+\sqrt{3}} \\
    \boxed{\lim_{x \to 0} \frac{1}{\sqrt{x+3}+\sqrt{3}} = 
    \frac{1}{\sqrt{0+3}+\sqrt{3}} 
    =\frac{1}{2\sqrt{3}}}
\end{multline*}
Multiplicando numerador e divisor por $\sqrt{3}$:
\begin{multline*}
    \lim_{x \to 0} \frac{1}{\sqrt{x+3}+\sqrt{3}} =\\ 
    \frac{1}{2\sqrt{3}} = 
    \frac{1(\sqrt{3})}{2\sqrt{3}(\sqrt{3})} = 
    \frac{\sqrt{3}}{2*3} \\ 
    \boxed{\lim_{x \to 0} \frac{1}{\sqrt{x+3}+\sqrt{3}} = \frac{\sqrt{3}}{6}}
\end{multline*}
\soln{2)}
Sabendo que $a^2 - b^2 = (a+b)(a-b)$:
\begin{multline*}
    \lim_{x \to 10} \frac{x^2-100}{x-10} \\
    \frac{x^2-10^2}{x-10} = \frac{(x+10)(x-10)}{x-10} = x+10 \\
    \boxed{\lim_{x\to10} x+10 = 20}
\end{multline*}
\soln{3)}
Sabendo que $a^2 - b^2 = (a+b)(a-b)$:
\begin{multline*}
    \lim_{x \to 7} \frac{x^2-49}{x-7}\\
    \frac{x^2 - 49}{x-7} = \frac{x^2 - 7^2}{x-7} = \frac{(x-7)(x+7)}{x-7} = x+7\\
    \boxed{\lim_{x \to 7} x+7 = 14}
\end{multline*}
\newpage
\soln{4)}
\begin{multline*}
    \lim_{x \to 0} \frac{3x^2 - 5x - 2}{x^2 + 3x - 10} \\
    \text{Sabendo que $ax^2 + bx + c = a(x-x1)(x-x2)$} \\
    x1, x2 \equiv \text{Raizes da função} \\
\end{multline*}
Utilizando a fórmula de báskara para achar as raízes da função:\\
Numerador ($3x^2 -5x -2$):
\begin{multline*}
    \{x1, x2\} = \frac{-(-5) \pm \sqrt{(-5)^2 - 4 * 3 * 2}}{2*3} \\
    = \frac{5 \pm \sqrt{25 + 24}}{6} = \frac{5 \pm \sqrt{49}}{6} = \frac{5 \pm 7}{6} \\
    \boxed{\{x1, x2\} = \biggl\{\frac{12}{6}, \frac{-2}{6}\biggr\} = \biggl\{2, -\frac{1}{3}\biggr\}}
\end{multline*}
Divisor: ($x^2 + 3x - 10$)
\begin{multline*}
    \{x1, x2\} = \frac{-3 \sqrt{3^2 - 4 * 1 * -10}}{2*1} \\= \frac{-3 \sqrt{3^2 - 4 * 1 * -10}}{2*1} = \frac{-3 \pm \sqrt{9 + 40}}{2} = \frac{-3 \pm 7}{2} \\
    \boxed{\{x1, x2\} = \biggl\{\frac{4}{2}, \frac{-10}{2}\biggr\} = \biggl\{2, -5\biggr\}}
\end{multline*}
Simplificando o divisor e numerador:
\begin{equation*}
    3x^2 - 5x - 2 = 3(x-2)(x+\frac{1}{3})
\end{equation*}
\begin{equation*}
    x^2 + 3x - 10 = 1(x-2)(x+5)
\end{equation*}
\begin{multline*}
    \lim_{x \to 2} \frac{3x^2 - 5x - 2}{x^2 + 3x - 10} \\ \frac{3x^2 - 5x - 2}{x^2 + 3x - 10} = \frac{3(x-2)(x+\frac{1}{3})}{(x-2)(x+5)} = \frac{3(x+\frac{1}{3})}{x+5} = \frac{3x+1}{x+5} = \frac{3x+1}{x+5} = \frac{3*2+1}{2+5} = \frac{7}{7} \\
    \boxed{\lim_{x \to 2} \frac{3x^2 - 5x - 2}{x^2 + 3x - 10} = 1}
\end{multline*}
\newpage
\soln{5)}
\begin{multline*}
    \lim_{x \to -1} \frac{x^2 + 3x + 2}{x^2 -1} \\
    \text{Sabendo que $ax^2 + bx + c = a(x-x1)(x-x2)$} \\
    x1, x2 \equiv \text{Raizes da função} \\
    \frac{-3 \pm \sqrt{-3^2-4*1*2}}{2*1} = \frac{-3 \pm \sqrt{9-8}}{2} = \frac{-3\pm1}{2}\\
    \{x1, x2\} = \biggl\{\frac{-2}{2}, \frac{-4}{2}\biggr\} = \biggl\{-1, -2\biggr\} \\
    x^2 + 3x + 2 = (x+1)(x+2)
\end{multline*}
Sabendo que $a^2 - b^2 = (a+b)(a-b)$
\[x^2 - 1 = x^2 - 1^2 = (x-1)(x+1)\]
Então:
\begin{multline*}
    \lim_{x \to -1} \frac{x^2 + 3x + 2}{x^2 - 1} \\ \frac{x^2 + 3x + 2}{x^2 - 1} = \frac{(x+1)(x+2)}{(x-1)(x+1)} = \frac{x+2}{x-1}
    = \frac{x+2}{x-1} = \frac{-1+2}{-1-1} = \frac{1}{-2} \\
    \boxed{\lim_{x \to -1} \frac{x^2 + 3x + 2}{x^2 - 1} = -\frac{1}{2}}
\end{multline*}
\end{document}