\documentclass{jhwhw}
\usepackage{pgfplots}
\pgfplotsset{compat=1.18}
\usepackage{cancel}
\usepackage{mathtools}
\usepackage{tikz}
\usepackage{amsmath}
\usepackage{multicol}
\usepackage[portuguese]{babel}
\title{Cálculo\\Lista 3 - Derivadas}
\author{Gabriel Vasconcelos Ferreira}
\begin{document}
\maketitle
\chapter{Regras de derivação}
\problem{
Calcular a derivada das funções, usando o formulário:
}
\begin{multicols}{3}[]
    \begin{enumerate}
        \item $y = -2x + 5$
        \item $y = \frac{1}{2}x^2 + \sqrt{x}$
        \item $y = 7x^2- 8x - 9$
        \item $y = -\frac{1}{3} x^2 + 5x + 7$
        \item $0.4 x^2 -6x -1$
        \item $(3x^2 - 4x)(6x+1)$
        \item $(1-x^2)(1+x^2)$
        \item $y = \sqrt{x}$
        \item $y = \sqrt[4]{x^2}$
        \item $y = \sqrt[9]{x}$
        \item $y = \frac{5}{x^3}$
        \item $y = \frac{4x}{x-1}$
    \end{enumerate}
\end{multicols}
\soln{1) $y = -2x + 5$}
\begin{align*}
    y\prime & = -2\cancel{x} + \cancel{5} \\
            & = \boxed{-2}                \\
\end{align*}
\soln{2) $y = \frac{1}{2}x^2 + \sqrt{x}$}
\begin{align*}
    y\prime & = \frac{1}{2}x^2 + \sqrt{x}                                    \\
            & = \frac{1}{\cancel{2}} \cdot \cancel{2}x + \frac{1}{2\sqrt{x}} \\
            & = \boxed{x + \frac{1}{2\sqrt{x}}}
\end{align*}

\soln{3) $y = 7x^2- 8x - 9$}
\begin{align*}
    y\prime & = 7x^2- 8x - 9                        \\
            & = 2\cdot7x - 8\cancel{x} - \cancel{9} \\
            & = \boxed{14x - 8}
\end{align*}

\soln{4) $y = -\frac{1}{3} x^2 + 5x + 7$}
\begin{align*}
    y\prime & = -\frac{1}{3} x^2 + 5\cancel{x} + \cancel{7} \\
            & = -\frac{1}{3} 2x + 5                         \\
            & = \boxed{-\frac{2}{3} x + 5}
\end{align*}
\soln{5) $y = 0.4 x^2 -6x -1$}
\begin{align*}
    y\prime & = 0.4 x^2 -6\cancel{x} \cancel{-1} \\
            & = 0.4\cdot2x -6                    \\
            & = \boxed{0.8x - 6}
\end{align*}
\soln{6) $y = (3x^2 - 4x)(6x+1)$}
Sabendo que:
\[
    y = u \cdot v
\]
\[
    y\prime = u\prime v + u v\prime
\]
\begin{multline*}
    y\prime = \overbrace{(3x^2 - 4x)}^{u}\overbrace{(6x+1)}^{v} \\
    = (3 \cdot 2x - 4)(6x+1) + 6(3x^2-4x) \\
    = (6x-4)(6x+1) + 18x^2 - 24x \\
    = 36x^2 + 6x -24x - 4 + 18x^2 - 24x \\
    = 36x^2 + 18x^2 +6x - 24x - 24x - 4 \\
    \boxed{y\prime = 54 - 42x - 4}
\end{multline*}
\soln{7)$(1-x^2)(1+x^2)$}
Sabendo que: \[
    y = u \cdot v\]\[
    y\prime = u\prime v + u v\prime
\]
\begin{multline*}
    y\prime = \overbrace{(1-x^2)}^{u} \cdot \overbrace{(1+x^2)}^{v} \\
    = -2x(1+x^2) + 2x(1-x^2) \\
    = \cancel{-2x} - 2x^3 + \cancel{2x} - 2x^3 \\
    \boxed{y\prime= -4x^3}
\end{multline*}
\soln{8) $y = \sqrt{x}$}
Sabendo que: \[
    \sqrt[n]{x} \implies y\prime = \frac{1}{n\sqrt[n]{x^{n-1}}}
\]
\begin{multline*}
    \\\boxed{y\prime = \frac{1}{2\sqrt{2x}}}\\
\end{multline*}
\soln{9) $y = \sqrt[4]{x^2}$}
Sabendo que: \[
    \sqrt[n]{x} \implies y\prime = \frac{1}{n\sqrt[n]{x^{n-1}}}
\]
\begin{align*}
    y\prime &= \frac{1}{4\sqrt[4]{(x^2)^{4}}}\\
    y\prime &= \frac{1}{4\sqrt[4]{x^6}}
\end{align*}
\newpage
\reset
\chapter{Regra da cadeia: derivada de funções compostas}
\problem{Determine a derivada de cada função, usando a regra da cadeia:}
\begin{enumerate}
    \item $y = (3x^3 + 7x^2 - 8x + 6)^5$
    \item $y = (x^2 -5)^10$
    \item $y = (5x+9)^5$
    \item $\sqrt[5]{x^3 + 6x - 1}$
\end{enumerate}
\newpage
\reset
\chapter{Derivada de funções trigonométricas}
\problem{Determine a derivada das funções:}
\begin{enumerate}
    \item $y = -\cos x$
    \item $y = x^2 \cdot x$
    \item $y = \frac{\cos x}{x}$
    \item $y = \sin (4x)$
    \item $y = \cos x \cdot \sin x$
\end{enumerate}
\newpage
\reset
\chapter{Aplicações da derivada}
\problem{Suponha que a equação do espaço $S$ (em metros) de um ponto material em função   do   tempo   (em   segundos)   é $S(t)=-3t2+18t+8$.   Determine   a velocidade instantânea do ponto material em $t = 2$ segundos.}
\problem{Suponhamos que daqui a xmeses a população de uma certa comunidade será $P(x)=x2+40x+3000$ habitantes.  Qual  a  taxa  de  variação  instantânea  da população em $x = 3$ meses?}
\problem{O volume de uma esfera de raio $r$ é dado por $V = \frac{4}{3}\pi r^3$. Qual é a taxa de variação instantânea do volume da esfera em relação ao raio para $r = 3$ cm?}
\problem{A área de um círculo de raio $r$ é dada por $A = \pi r^2$. No decorrer de uma experiência, derrama-se um líquido sobre uma superfície plana de vidro. Se o líquido vertido recobre uma região circular e o raio desta região aumenta uniformemente, qual será a taxa de crescimento da área ocupada pelo líquido, em relação à variação do raio, quando o raio for igual a 5cm?}
\problem{}
\end{document}