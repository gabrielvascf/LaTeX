\documentclass{jhwhw}
\usepackage{pgfplots}
\pgfplotsset{compat=1.18}
\usepackage{cancel}
\usepackage{mathtools}
\usepackage{tikz}
\usepackage{amsmath}
\usepackage{multicol}
\usepackage[portuguese]{babel}
\title{Cálculo\\Lista 3 - Derivadas}
\author{Gabriel Vasconcelos Ferreira}
\begin{document}
\maketitle
\chapter{Regras de derivação}
\problem{
Calcular a derivada das funções, usando o formulário:
}
\begin{multicols}{3}[]
    \begin{enumerate}
        \item $\displaystyle y = -2x + 5$
        \item $\displaystyle y = \frac{1}{2}x^2 + \sqrt{x}$
        \item $\displaystyle y = 7x^2- 8x - 9$
        \item $\displaystyle y = -\frac{1}{3} x^2 + 5x + 7$
        \item $\displaystyle 0.4 x^2 -6x -1$
        \item $\displaystyle (3x^2 - 4x)(6x+1)$
        \item $\displaystyle (1-x^2)(1+x^2)$
        \item $\displaystyle y = \sqrt{x}$
        \item $\displaystyle y = \sqrt[4]{x^2}$
        \item $\displaystyle y = \sqrt[9]{x}$
        \item $\displaystyle y = \frac{5}{x^3}$
        \item $\displaystyle y = \frac{4x}{x-1}$
    \end{enumerate}
\end{multicols}
\soln{1) $\displaystyle y = -2x + 5 \implies y' = -2\cancel{x} + \cancel{5} = \boxed{-2}$}
\soln{2) $\displaystyle y = \frac{1}{2}x^2 + \sqrt{x} \implies y' = \frac{1}{2} \cdot \cancelto{2x}{x^2} + \frac{1}{2\sqrt{x}} = \frac{1}{\cancel{2}} \cdot \cancel{2}x + \frac{1}{2\sqrt{x}} = \boxed{x + \frac{1}{2\sqrt{x}}}$}
\soln{3) $\displaystyle y = 7x^2- 8x - 9 \implies y' = 2\cdot7x - 8\cancel{x} - \cancel{9} = \boxed{14x - 8}$}
\soln{4) $\displaystyle y = -\frac{1}{3} x^2 + 5x + 7 \implies y' = -\frac{1}{3} x^2 + 5\cancel{x} + \cancel{7} = -\frac{1}{3} 2x + 5 = \boxed{-\frac{2}{3} x + 5}$}
\soln{5) $y = 0.4 x^2 -6x -1 \implies y' = 0.4 x^2 -6\cancel{x} \cancel{-1} = 0.4\cdot2x -6 = \boxed{0.8x - 6}$}
\newpage
\soln{6) $y = (3x^2 - 4x)(6x+1)$}
Sabendo que:
\[
    y = u \cdot v
\]
\[
    y' = u' v + u v'
\]
Temos:
\begin{multline*}
    y' = \overbrace{(3x^2 - 4x)}^{u}\overbrace{(6x+1)}^{v} \\
    = (3 \cdot 2x - 4)(6x+1) + 6(3x^2-4x) \\
    = (6x-4)(6x+1) + 18x^2 - 24x \\
    = 36x^2 + 6x -24x - 4 + 18x^2 - 24x \\
    = 36x^2 + 18x^2 +6x - 24x - 24x - 4 \\
    \boxed{y' = 54 - 42x - 4}
\end{multline*}
\soln{7)$(1-x^2)(1+x^2)$}
Sabendo que: \[
    y = u \cdot v\]\[
    y' = u' v + u v'
\]
Temos:
\begin{multline*}
    y' = \overbrace{(1-x^2)}^{u} \cdot \overbrace{(1+x^2)}^{v} \\
    = -2x(1+x^2) + 2x(1-x^2) \\
    = \cancel{-2x} - 2x^3 + \cancel{2x} - 2x^3 \\
    \boxed{y'= -4x^3}
\end{multline*}
\soln{8) $y = \sqrt{x}$}
Sabendo que: \[
    \sqrt[n]{x} \implies y' = \frac{1}{n\sqrt[n]{x^{n-1}}}
\]
Temos:
\begin{multline*}
    \\\boxed{y' = \frac{1}{2\sqrt{2x}}}\\
\end{multline*}
\soln{9) $y = \sqrt[4]{x^2}$}
Sabendo que: \[
    \sqrt[n]{x} \implies y' = \frac{1}{n\sqrt[n]{x^{n-1}}}
\]
Temos:
\begin{align*}
    y' & = \frac{1}{4\sqrt[4]{(x^2)^{3}}} \\
    y' & = \frac{1}{4\sqrt[4]{x^6}}
\end{align*}
\soln{10) $y = \sqrt[9]{x}$}
\begin{align*}
    y = \sqrt[9]{x} \implies y' = \frac{1}{9\sqrt[9]{x^{9-1}}} = \frac{1}{9\sqrt[9]{x^{8}}}
\end{align*}
\soln{11) $y = \frac{5}{x^3}$}
Sabendo que \[
    y = \frac{u}{v} \implies y' = \frac{u' v - v' u}{v^2}
\] 
Temos:
\begin{multline*}
    y = \frac{5}{x^3} \implies \begin{cases}
        u = 5 \\
        v = x^3
    \end{cases}\\
    y' = \frac{0 \cdot x^3 - 5 \cdot 3x^2}{(x^3)^2}
    = \frac{-15x^2}{x^6}
    = -15 \frac{x^2}{x^6} = -15 x^{-4} = -15 \frac{1}{x^4} \\ \boxed{y' = -\frac{15}{x^4}}
\end{multline*}
\soln{12) $y = \frac{4x}{x-1}$}
Sabendo que \[
    y = \frac{u}{v} \implies y' = \frac{u' v - v' u}{v^2}
\] 
Temos:
\begin{multline*}
    y = \frac{4x}{x-1} \implies \begin{cases}
        u = 4x \\
        v = x-1
    \end{cases} \implies y' = \frac{4 \cdot (x-1) - 4x \cdot (1)}{(x-1)^2}\\
    y' = \frac{4 \cdot (x-1) - 4x \cdot (1)}{(x-1)^2} = \frac{\cancel{4x}-4-\cancel{4x}}{x^2-2x-1} \\ \boxed{y' = \frac{-4}{x^2-2x-1}}
\end{multline*}
\newpage
\reset
\chapter{Regra da cadeia: derivada de funções compostas}
\problem{Determine a derivada de cada função, usando a regra da cadeia:}
\begin{enumerate}
    \item $y = (3x^3 + 7x^2 - 8x + 6)^5$
    \item $y = (x^2 -5)^{10}$
    \item $y = (5x+9)^5$
    \item $\sqrt[5]{x^3 + 6x - 1}$
\end{enumerate}
\soln{1) $y = (3x^3 + 7x^2 - 8x + 6)^5$}
\begin{align*}
    y & = (\underbrace{3x^3 + 7x^2 - 8x + 6}_{u})^5 \implies
    \begin{cases}
        u = 3x^3 + 7x^2 - 8x + 6 & \implies u'^{}_x = 9x^2 + 14x - 8 \\
        y = u^5                  & \implies y'^{}_u = 5u^4
    \end{cases}
\end{align*}
Pela regra da cadeia:
\begin{multline*}
    y'^{}_x = u'^{}_x \cdot y'^{}_u \implies y'^{}_x = 5u^4(9x^2 + 14x - 8) \\ \boxed{y' = 5(3x^3 + 7x^2 - 8x + 6)^4(9x^2 + 14x - 8)}
\end{multline*}
\soln{2) $y = (x^2 -5)^{10}$}
\begin{align*}
    y & = (\underbrace{x^2 -5}_{u})^{10} \implies
    \begin{cases}
        u = x^2 -5 & \implies u'^{}_x = 2x    \\
        y = u^{10} & \implies y'^{}_u = 10u^9
    \end{cases}
\end{align*}
Pela regra da cadeia:
\begin{multline*}
    y'^{}_x = u'^{}_x \cdot y'^{}_u \implies y'^{}_x = 2x \cdot 10u^9 = 2x \cdot 10(x^2-5)^9 \\ \boxed{y' = 20x\cdot(x^2-5)^9}
\end{multline*}
\soln{3) $y = (5x+9)^5$}
\begin{align*}
    y & = (\underbrace{5x+9}_{u})^{5} \implies
    \begin{cases}
        u = 5x+9  & \implies u'^{}_x = 5    \\
        y = u^{5} & \implies y'^{}_u = 5u^4
    \end{cases}
\end{align*}
Pela regra da cadeia:
\begin{multline*}
    y'^{}_x = u'^{}_x \cdot y'^{}_u \implies y'^{}_x = 5(5u^4) = 5(5(5x+9)^4)\\ \boxed{y' = 25 (5x+9)^4}
\end{multline*}
\soln{4) $\sqrt[5]{x^3 + 6x - 1}$}
\begin{align*}
    y & = \sqrt[5]{\underbrace{x^3 + 6x - 1}_{u}} \implies
    \begin{cases}
        u = x^3 + 6x - 1 & \implies u'^{}_x = 3x^2 + 6                 \\
        y = \sqrt[5]{u}  & \implies y'^{}_u = \frac{1}{5\sqrt[5]{u^4}}
    \end{cases}
\end{align*}
Pela regra da cadeia:
\begin{multline*}
    y'^{}_x = u'^{}_x \cdot y'^{}_u \implies y'^{}_x = (3x^2+6)\left(\frac{1}{5\sqrt[5]{u^4}}\right)  = \frac{3x^2+6}{5\sqrt[5]{u^4}} \\  \boxed{y'^{}_x = \frac{3x^2+6}{5\sqrt[5]{(x^3 + 6x - 1)^4}}}
\end{multline*}
\newpage
\reset
\chapter{Derivada de funções trigonométricas}
\problem{Determine a derivada das funções:}
\begin{enumerate}
    \item \(\displaystyle y = -\cos x\)
    \item \(\displaystyle y = x^2 \cdot x\)
    \item \(\displaystyle y = \frac{\cos x}{x}\)
    \item \(\displaystyle y = \sin (4x)\)
    \item \(\displaystyle y = \cos x \cdot \sin x\)
\end{enumerate}
\soln{1) $y = -\cos x \implies y' = -1(-\sin x) = \boxed{\sin x}$}
\soln{2) $y = x^2 \cdot x \implies \boxed{y' = 3x^2}$}
\soln{3) $\displaystyle y = \frac{\cos x}{x} \implies y' = \frac{-\sin x \cdot x - \cos x}{x^2} = \boxed{\frac{-x \sin x - \cos x}{x^2}}$}
\soln{4) $y = \sin (\underbrace{4x}_u) \implies y' = u' \cos u \implies \boxed{y' = 4 \cos(4x)}$}
\soln{5) $y = \underbrace{\cos x}_u \cdot \underbrace{\sin x}_v \implies y' = u'v + uv' \implies y' = \boxed{-sin^2x + \cos^2x}$}
\reset
\chapter{Aplicações da derivada}
\problem{Suponha que a equação do espaço $S$ (em metros) de um ponto material em função   do   tempo   (em   segundos)   é $S(t)=-3t2+18t+8$.   Determine   a velocidade instantânea do ponto material em $t = 2$ segundos.}
\problem{Suponhamos que daqui a $x$ meses a população de uma certa comunidade será $P(x)=x^2+40x+3000$ habitantes.  Qual  a  taxa  de  variação  instantânea  da população em $x = 3$ meses?}
\problem{O volume de uma esfera de raio $r$ é dado por $V = \frac{4}{3}\pi r^3$. Qual é a taxa de variação instantânea do volume da esfera em relação ao raio para $r = 3$ cm?}
\problem{A área de um círculo de raio $r$ é dada por $A = \pi r^2$. No decorrer de uma experiência, derrama-se um líquido sobre uma superfície plana de vidro. Se o líquido vertido recobre uma região circular e o raio desta região aumenta uniformemente, qual será a taxa de crescimento da área ocupada pelo líquido, em relação à variação do raio, quando o raio for igual a 5cm?}
\newpage
\soln{Problema 1:}
\begin{multline*}
    S(t) = -3\cancelto{2t}{t^2} + 18\cancel{t} + \cancel{8} \implies S'(t) = -3(2t) + 18\\\boxed{S(t)' = -6t + 18}\\
    S'(2) = -6(2) + 18 = -12 + 18 \\ = \boxed{6 \text{ metros por segundo.}}
\end{multline*}
\soln{Problema 2:}
\begin{multline*}
    P(x) = \cancelto{2x}{x^2} + 40\cancel{x} + \cancel{3000} \implies \boxed{P'(x) = 2x + 40}\\P'(3) = 2(3)+ 40 \\ = \boxed{46 \text{ habitantes por mês.}}
\end{multline*}
\soln{Problema 3:}
\begin{multline*}
    V(r) = \frac{4}{3} \pi \cancelto{3r^2}{r^3} \implies \boxed{V'(r) = \frac{4}{3}\pi 3r^2}\\
    V'(3) = \frac{4}{\cancel{3}}\pi \cancel{3}(3^2) = 4\pi9 \\= \boxed{36\pi \frac{\operatorname{cm}^2}{\operatorname{cm}}}
\end{multline*}
\soln{Problema 4:}
\begin{multline*}
    A(r) = \pi \cancelto{2r}{r^2} \implies A'(r) = 2\pi r \\
    A'(5) = 2\pi(5) \\ = \boxed{10\pi \frac{\operatorname{cm}^2}{\operatorname{cm}}}
\end{multline*}
\end{document}