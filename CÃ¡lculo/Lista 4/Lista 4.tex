\documentclass{jhwhw}
\usepackage{pgfplots}
\usepackage{cancel}
\usepackage{mathtools}
\usepackage{tikz}
\usepackage{amsmath}
\usepackage{multicol}
\usepackage[portuguese]{babel}

\pgfplotsset{compat=1.18}
\usepgfplotslibrary{fillbetween}
\usetikzlibrary{patterns}

\newcommand{\dx}{\mathrm{dx}}
\newcommand{\du}{\mathrm{du}}


\title{Cálculo\\Lista 4 --- Integrais}
\author{Gabriel Vasconcelos Ferreira}
\begin{document}
\maketitle
\chapter{Integrais imediatas / quase imediatas}
\anyproblem{
    Resolva as integrais imediatas ou quase imediatas:
}
\begin{multicols}{2}[]
    \begin{enumerate}
        \item $\displaystyle \int (3x^2 - 2x + 4) \dx$
        \item $\displaystyle \int \left( \frac{x^3}{2} -1 \right) \dx $
        \item $\displaystyle \int \frac{1-x}{2} \dx$
        \item $\displaystyle \int \left(\frac{1}{3}x^2 - \frac{1}{2}x - 3\right) \dx$
    \end{enumerate}
\end{multicols}
\soln{1)
\begin{multline*}
    \int (3x^2 - 2x + 4 \dx) \implies \cancelto{\frac{3x^3}{3}}{\int 3x^2} - \cancelto{\frac{2x^2}{2x}}{\int 2} + \cancelto{4x}{\int 4} = 
    \frac{\cancel{3}x^3}{\cancel{3}} - \frac{\cancel{2}x^2}{\cancel{2}} + 4x = 
    \\ \boxed{\int (3x^2 - 2x + 4 \dx) = x^3 - x^2 + 4x + C}
\end{multline*}
}

\soln{2)
\begin{multline*}
    \int \left( \frac{x^3}{2} -1 \right) \dx  \implies 
    \frac{1}{2} \cancelto{\frac{x^4}{4}}{\int x^3 \dx} - \cancelto{x}{\int 1 \dx} = 
    \frac{1}{2}\frac{x^4}{4} - x + C\\ \boxed{\int \left( \frac{x^3}{2} -1 \right) = \frac{x^4}{8} - x + C}
\end{multline*}
}

\soln{3)
    \begin{multline*}
        \int \frac{1-x}{2} \dx \implies 
        \int \frac{1}{2} \dx - \cancelto{\frac{1}{2}x}{\int \frac{x}{2}} \dx = 
        \frac{x}{2} - \frac{1}{2} \cancelto{\frac{x^2}{2}}{\int x} \dx = \frac{x}{2} - \frac{1}{2} \frac{x^2}{2} + C \\
        \boxed{\int \frac{1-x}{2} \dx = \frac{x}{2} - \frac{x^2}{4} + C}
    \end{multline*}
}

\soln{4)
    \begin{multline*}
        \int \left( \frac{1}{3} x^2 - \frac{1}{2}x - 3 \right) \dx \implies
        \frac{1}{3} \int x^2 \dx - \frac{1}{2} \int x \dx - \int 3 \dx \\
        = \frac{1}{3} \frac{x^3}{3} - \frac{1}{2} \frac{x^2}{2} - 3x + C \\
        \boxed{\int \left( \frac{1}{3} x^2 - \frac{1}{2}x - 3 \right) \dx = \frac{x^3}{9} - \frac{x^2}{4} - 3x + C}
    \end{multline*}
}

\reset{}
\chapter{Integral por substituição}
\anyproblem{Resolva as integrais por substituição:}
\begin{multicols}{2}[]
    \begin{enumerate}
        \item $\displaystyle \int \frac{-6x - 5}{-3x^2 - 5x - 2} \dx$
        \item $\displaystyle \int \frac{3x -1}{3x^2 - 2x} \dx $
        \item $\displaystyle \int {(x^2 - 5)}^3x \dx$
        \item $\displaystyle \int \frac{\dx}{{(5-3x)}^2}$
    \end{enumerate}
\end{multicols}

\soln{1)$$
    \int \frac{-6x - 5}{-3x^2 - 5x - 2} \dx \implies \int \underbrace{\frac{1}{-3x^2 -5x - 2}}_u \underbrace{(-6x - 5) \dx}_{\du}
$$
\[u = -3x^2 -5x - 2\]
\[\frac{\du}{\dx} = -6x - 5 \implies \du = (-6x-5) \dx\]
\[\int \frac{1}{u} \du \implies \ln u + C \implies \boxed{\ln |-3x^2 -5x - 2| + C}\]
}

\soln{2) \[
    \int \frac{3x -1}{3x^2 - 2x} \dx \implies \int \frac{1}{\underbrace{3x^2 - 2x}_u} \underbrace{(3x-1)\dx}_\du
\]
\[
    u = 3x^2-2x
\]
\begin{multline*}
    \frac{\du}{\dx} = 6x - 2 \implies \\ 
    \du = (6x-2)\dx =\\ 
    \du = 2(3x-1)\dx\\
    \boxed{\frac{\du}{2} = (3x-1)\dx}
\end{multline*}
\begin{multline*}
    \int \frac{1}{u} \frac{\du}{2} =\\
    \int \frac{1}{u} \frac{\du}{2} = \int \frac{1}{u} \frac{1}{2}\du \implies \frac{1}{2} \int \frac{1}{u} \du \\
    \frac{1}{2} \ln |u| + C \implies \\
    \boxed{\int \frac{3x -1}{3x^2 - 2x} \dx = \frac{1}{2} \ln |3x^2-2x| + C}
\end{multline*}
}

\soln{3)
\[\int {\underbrace{(x^2 - 5)}_u}^3\underbrace{x \dx}_\du\]
\[
    u = x^2-5 \implies u' = 2x
\]
\[
    \frac{\du}{\dx} = 2x \implies \du = 2x\dx \implies \frac{\du}{2} = x\dx
\]
\begin{multline*}
    \int (x^2-5)^3 x \dx \implies \\ 
    \int (u)^3 \frac{\du}{2} \implies 
    \int (u)^3 \du \implies
    \frac{1}{2} \int u^3 \du \implies
    \frac{1}{2} \frac{u^4}{4} + C = \frac{u^4}{8} + C \implies \\
    \boxed{\int (x^2-5)^3 x \dx = \frac{(x^2-5)^4}{8} + C}
\end{multline*}
}

\soln{4) 
\[
    \int \frac{\dx}{{(5-3x)}^2} = \int \frac{1}{{\underbrace{(5-3x)}_u}^2} \underbrace{\dx}_\du
\]
\[
    u = -3x+5 \implies u' = -3
\]
\[
    \frac{\du}{\dx} = -3 \implies \du = -3\dx \implies -\frac{\du}{3} = \dx
\]
\begin{multline*}
        \int \frac{1}{u^2} \left(-\frac{du}{3}\right) \implies \\
        \int u^{-2} \left(-\frac{1}{3}\du\right) = 
        -\frac{1}{3} \int u^{-2} \du \implies 
        -\frac{1}{3} \frac{u^{-1}}{-1} + C = \frac{1}{3} \frac{1}{-3x+5} + C = \frac{1}{3(-3x+5)} + C\\
        \boxed{\int \frac{\dx}{{(5-3x)}^2}  = \frac{1}{-9x+15} + C}
\end{multline*}
}

\reset{}
\chapter{Integrais definidas --- aplicações}
\anyproblem{1) Calcule a área entre os gráficos de $y = x + 2$ e $y = x^2$}
\anyproblem{2) Calcule a área limitada pela curva $y = -x^2 + 5x$ e pelo \emph{eixo} $x$.}
\anyproblem{3) Calcula a área sob o (abaixo do) gráfico da função $y = x$, de $x = 0$ a $x = 3$.}
\anyproblem{4) Determine o volume do sólido de revolução gerado pela rotação \emph{em torno do eixo} $x$, da região limitada por $y = 3x+1$, $x = 0$, $x = 3$ e $y = 0$}
\anyproblem{5) Determine o volume do sólido de revolução gerado pela rotação \emph{em torno do eixo} $x$, da região limitada por $y = x^2 + 1$, $x = 1$, $x = 3$ e $y = 0$.}

\soln{1) Calcule a área entre os gráficos de $y = x + 2$ e $y = x^2$}
\part{Gráfico:}
\begin{center}
	\begin{tikzpicture}
		\begin{axis}[
				xlabel={$x$},
				ylabel={$y$},
				axis lines=middle,
				grid=both,
				minor tick num = 1,
				enlargelimits={abs=0.25},
                domain = -3:3
			]
			\addplot[red,thick,name path=A]{x*x};
			\addplot[blue,thick,name path=B]{x+2};
            \addplot[pattern=north west lines, pattern color=gray!50] fill between[of=A and B, soft clip={domain=-1:2}];
		\end{axis}
	\end{tikzpicture}
\end{center}
\part{Área}
\[f(x) = x^2 \implies F(x) = \int f(x) \dx = \int x^2 \dx = \boxed{\frac{x^3}{3} + C}\]
\[g(x) = x+2 \implies G(X) = \int g(x) \dx = \int x+2 \dx = \boxed{\frac{x^2}{2} + 2x + C}\]
Pontos de interseção:
\begin{multline*}
    f(x) = g(x) \implies x^2 = x+2 \implies 
    \underbrace{x^2}_a \underbrace{-x}_b \underbrace{-2}_c = 0 \\
    \frac{-b \pm \sqrt{b^2-4ac}}{2a} \implies \frac{1 \pm \sqrt{-1^2 - 4(1)(-2)}}{2} = \\
    \frac{1 \pm \sqrt{1 + 8}}{2} = 
    \frac{1 \pm \sqrt{9}}{2} = \\
    \frac{1 \pm 3}{2} = 
    -\frac{2}{2}, \frac{4}{2}\\
    \boxed{\{a, b\} = -1, 2} 
\end{multline*}
\begin{align*} 
    A & = \int_{a}^{b} [f(x) - g(x)] \dx \\
    A & = [F(x) - G(x)]_a^b \\
    A & = \left[\frac{x^3}{3} - \frac{x^2}{2}\right]_{-1}^2 \\
    \shortintertext{Como $G(X) \geq F(X)$ em $[-1, 2]$}
    A & = [G(2) - F(2)] - [G(-1) - F(-1)] 
\end{align*}
\end{document}