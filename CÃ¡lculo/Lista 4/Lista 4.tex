\documentclass{jhwhw}
\usepackage{pgfplots}
\pgfplotsset{compat=1.18}
\usepackage{cancel}
\usepackage{mathtools}
\usepackage{tikz}
\usepackage{amsmath}
\usepackage{multicol}
\usepackage[portuguese]{babel}

\newcommand{\dx}{\mathrm{dx}}
\newcommand{\du}{\mathrm{du}}


\title{Cálculo\\Lista 4 --- Integrais}
\author{Gabriel Vasconcelos Ferreira}
\begin{document}
\maketitle
\chapter{Integrais imediatas / quase imediatas}
\anyproblem{
    Resolva as integrais imediatas ou quase imediatas:
}
\begin{multicols}{2}[]
    \begin{enumerate}
        \item $\displaystyle \int (3x^2 - 2x + 4) \dx$
        \item $\displaystyle \int \left( \frac{x^3}{2} -1 \right) \dx $
        \item $\displaystyle \int \frac{1-x}{2} \dx$
        \item $\displaystyle \int \left(\frac{1}{3}x^2 - \frac{1}{2}x - 3\right) \dx$
    \end{enumerate}
\end{multicols}
\soln{1)
\begin{multline*}
    \int (3x^2 - 2x + 4 \dx) \implies \cancelto{\frac{3x^3}{3}}{\int 3x^2} - \cancelto{\frac{2x^2}{2x}}{\int 2} + \cancelto{4x}{\int 4} = 
    \frac{\cancel{3}x^3}{\cancel{3}} - \frac{\cancel{2}x^2}{\cancel{2}} + 4x = 
    \\ \boxed{\int (3x^2 - 2x + 4 \dx) = x^3 - x^2 + 4x + C}
\end{multline*}
}

\soln{2)
\begin{multline*}
    \int \left( \frac{x^3}{2} -1 \right) \dx  \implies 
    \frac{1}{2} \cancelto{\frac{x^4}{4}}{\int x^3 \dx} - \cancelto{x}{\int 1 \dx} = 
    \frac{1}{2}\frac{x^4}{4} - x + C\\ \boxed{\int \left( \frac{x^3}{2} -1 \right) = \frac{x^4}{8} - x + C}
\end{multline*}
}

\soln{3)
    \begin{multline*}
        \int \frac{1-x}{2} \dx \implies 
        \int \frac{1}{2} \dx - \cancelto{\frac{1}{2}x}{\int \frac{x}{2}} \dx = 
        \frac{x}{2} - \frac{1}{2} \cancelto{\frac{x^2}{2}}{\int x} \dx = \frac{x}{2} - \frac{1}{2} \frac{x^2}{2} + C \\
        \boxed{\int \frac{1-x}{2} \dx = \frac{x}{2} - \frac{x^2}{4} + C}
    \end{multline*}
}

\soln{4)
    \begin{multline*}
        \int \left( \frac{1}{3} x^2 - \frac{1}{2}x - 3 \right) \dx \implies
        \frac{1}{3} \int x^2 \dx - \frac{1}{2} \int x \dx - \int 3 \dx \\
        = \frac{1}{3} \frac{x^3}{3} - \frac{1}{2} \frac{x^2}{2} - 3x + C \\
        \boxed{\int \left( \frac{1}{3} x^2 - \frac{1}{2}x - 3 \right) \dx = \frac{x^3}{9} - \frac{x^2}{4} - 3x + C}
    \end{multline*}
}

\reset{}
\chapter{Integral por substituição}
\anyproblem{Resolva as integrais por substituição:}
\begin{multicols}{2}[]
    \begin{enumerate}
        \item $\displaystyle \int \frac{-6x - 5}{-3x^2 - 5x - 2} \dx$
        \item $\displaystyle \int \frac{3x -1}{3x^2 - 2x} \dx $
        \item $\displaystyle \int {(x^2 - 5)}^3x \dx$
        \item $\displaystyle \int \frac{\dx}{{(5-3x)}^2}$
    \end{enumerate}
\end{multicols}

\soln{1)$$
    \int \frac{-6x - 5}{-3x^2 - 5x - 2} \dx \implies \int \underbrace{\frac{1}{-3x^2 -5x - 2}}_u \underbrace{(-6x - 5) \dx}_{\du}
$$
\[u = -3x^2 -5x - 2\]
\[\frac{\du}{\dx} = -6x - 5 \implies \du = (-6x-5) \dx\]
\[\int \frac{1}{u} \du \implies \ln u + C \implies \boxed{\ln (-3x^2 -5x - 2) + C}\]
}

\soln{2) \[
    \int \frac{3x -1}{3x^2 - 2x} \dx \implies \int \frac{1}{\underbrace{3x^2 - 2x}_u} \underbrace{(3x-1)\dx}_\du
\]
\[
    u = 3x^2 - 2x
\]
\[
    \frac{\du}{\dx} = 6x - 2 \implies \du = (6x-2)\dx
\]
}

\reset{}
\chapter{Integrais definidas --- aplicações}
\anyproblem{1) Calcule a área entre os gráficos de $y = x + 2$ e $y = x^2$}
\anyproblem{2) Calcule a área limitada pela curva $y = -x^2 + 5x$ e pelo \emph{eixo} $x$.}
\anyproblem{3) Calcula a área sob o (abaixo do) gráfico da função $y = x$, de $x = 0$ a $x = 3$.}
\anyproblem{4) Determine o volume do sólido de revolução gerado pela rotação \emph{em torno do eixo} $x$, da região limitada por $y = 3x+1$, $x = 0$, $x = 3$ e $y = 0$}
\anyproblem{5) Determine o volume do sólido de revolução gerado pela rotação \emph{em torno do eixo} $x$, da região limitada por $y = x^2 + 1$, $x = 1$, $x = 3$ e $y = 0$.}
\end{document}