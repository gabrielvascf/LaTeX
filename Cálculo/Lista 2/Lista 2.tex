\documentclass{jhwhw}
\usepackage{pgfplots}
\pgfplotsset{compat=1.18}
\usepackage{mathtools}
\usepackage{tikz}
\usepackage[portuguese]{babel}
\title{Cálculo\\Lista 2 - Limites}
\author{Gabriel Vasconcelos Ferreira}
\begin{document}
\maketitle
\chapter{Noção Intuitiva / Limites laterais}
\problem{
Dadas as funções e seus respectivos gráficos:
}
\begin{enumerate}
    \item calcule os limites laterais da $f(x)$
    \item compare os limites laterais e verifique se eles são iguais ou diferentes
    \item conclua se existe o limite da função e se existir, indique qual é o seu valor
\end{enumerate}
\begin{enumerate}
    \item [1)]$$
              \lim _{x \to 0} f(x), \operatorname{para} f(x)= \begin{cases}
                  x^2 & \text {, se } x \leq 0 \\ 1+x^2 & \text {, se } x>0
              \end{cases}$$
    \item [2)]$$
              \lim _{x \to 2} f(x), \operatorname{para} f(x)= \begin{cases}
                  x^2 & \text {, se } x \geq 2 \\ x+1 & \text {, se } x<2
              \end{cases}
          $$
    \item [3)]$$
              \lim _{x \to 1} f(x), \operatorname{para} f(x)= \begin{cases}
                  3x+1 & \text {, se } x \neq 1 \\ 0 & \text {, se } x=1
              \end{cases}
          $$
    \item [4)]$$
              \lim _{x \to 1} f(x), \operatorname{para} f(x)= \begin{cases}
                  3x+1 & \text {, se } x \neq 1 \\ 0 & \text {, se } x=1
              \end{cases}
          $$
\end{enumerate}
\newpage
\soln{1)$$
        \lim _{x \to 0} f(x), \operatorname{para} f(x)= \begin{cases}
            x^2 & \text {, se } x \leq 0 \\ 1+x^2 & \text {, se } x>0
        \end{cases}$$}
\begin{align*}
    \lim _{x \rightarrow 0^-} & f(x) = x^2 & \lim_{x\to0^+} & f(x) = 1+x^2   \\
                              & f(0) = 0^2 &                & f(0) = 1 + 0^2 \\
                              & f(0) = 0   &                & f(0) = 1       \\
    \lim _{x \rightarrow 0^-} & f(x) = 0   & \lim_{x\to0^+} & f(x) = 1
\end{align*}
\[
    \lim _{x \rightarrow 0^-} f(x) \neq \lim _{x \rightarrow 0^+} f(x)
\]
\soln{2)$$
        \lim _{x \to 2} f(x), \operatorname{para} f(x)= \begin{cases}
            x^2 & \text {, se } x \geq 2 \\ x+1 & \text {, se } x<2
        \end{cases}
    $$}
\begin{align*}
    \lim_{x \to 2^-} & f(x) = x^2 & \lim_{x\to2^+} & f(x) = x + 1 \\
                     & f(2) = 2^2 &                & f(2) = 2 + 1 \\
                     & f(2) = 4   &                & f(2) = 3     \\
    \lim_{x \to 2^-} & f(x) = 4   & \lim_{x\to2^+} & f(x) = 3
\end{align*}
\[
    \lim_{x \to 2^-} f(x) \neq \lim_{x\to2^+} f(x)
\]
\soln{3)$$
        \lim _{x \to 1} f(x), \operatorname{para} f(x)= \begin{cases}
            3x+1 & \text {, se } x \neq 1 \\ 0 & \text {, se } x=1
        \end{cases}
    $$}
\begin{align*}
    \lim_{x \to 1} f(x) & = 3*1+1 \\
                        & = 3 + 1 \\
                        & = 4     \\
    \lim_{x \to 1}      & = 4
\end{align*}
\[\]
\end{document}