\documentclass{jhwhw}
\usepackage{pgfplots}
\usepackage{cancel}
\usepackage{mathtools}
\usepackage{tikz}
\usepackage{amsmath}
\usepackage{multicol}
\usepackage{enumitem}
\usepackage[portuguese]{babel}

\pgfplotsset{compat=1.18}
\usepgfplotslibrary{fillbetween}
\usetikzlibrary{patterns}

\newcommand{\dx}{\mathrm{dx}}
\newcommand{\du}{\mathrm{du}}
\newcommand{\fr}{\mathrm{Fr}}

\begin{document}
\author{Gabriel Vasconcelos Ferreira}
\title{%
    Estatística\\Lista 1 --- Distribuição de frequências: tabelas 
}
\maketitle    
\anyproblem{1- Com os dados brutos abaixo, faça o que se pede a seguir.}
\begin{center}
    \textbf{Previsão de temperatura máxima, em graus Celsius, em São Paulo, diariamente, pelo período de 14 dias, de 07 a 21 de fevereiro de 2025.}
    \begin{tabular}{llllllllllllll}
        \hline 31 & 27 & 29 & 30 & 31 & 32 & 31 & 30 & 30 & 31 & 31 & 32 & 31 & 31 \\
        \hline
    \end{tabular}\\
    \textbf{Fonte:} meteoblue --- A Windy.com Company. Disponível em: \url{https://www.meteoblue.com/pt/tempo/14-dias/s\%c3\%a3o-paulo\_brasil\_3448439}. Acesso em 07 fev. 2025.
\end{center}
\begin{enumerate}[label=\alph*)]
    \item Faça a \textbf{Tabela de Distribuição de Frequências} (4 tipos de frequências) para \textbf{variável discreta}, ou seja, sem agrupar os dados. Não esqueça de colocar o título e a fonte da tabela.
    \item Qual é o número de vezes que a previsão de temperatura máxima foi de 30ºC?\\Qual é a porcentagem correspondente?
    \item Qual é o número de vezes que a previsão de temperatura máxima foi de 27 a 30ºC?\\Qual é a porcentagem correspondente?
    \item Qual é o número de vezes que a previsão de temperatura máxima foi de 32ºC?\\Qual é a porcentagem correspondente?
\end{enumerate}

\newpage
\soln{Questão 1:}
\soln{a) }
\begin{align*}
    i &= 1 + 3.3 \cdot \log{n}\\
    i &= 1 + 3.3 \cdot \log{14}\\
    i &= 4.7822\dots\\
    i &= \boxed{5}
\end{align*}
Rol $\rightarrow$ 
    \boxed{
        \begin{tabular}{c c c c}
            27 & 30 & 31 & 32 \\
            29 & 31 & 31 & 32 \\
            30 & 31 & 31 &    \\
            30 & 31 & 31 & 
        \end{tabular}
    }

\[
    \fr_i = \frac{fi}{\Sigma fi} 
\]
\begin{align*}
    & \fr_1 = \frac{1}{14} \times 100 = 7.14   & & \fr_4 = \frac{7}{14} \times 100 = 50 \\
    & \fr_2 = \frac{2}{14} \times 100 = 14.28  & & \fr_5 = \frac{2}{14} \times 100 = 14.28 \\
    & \fr_3 = \frac{3}{14} \times 100 = 21.42  & &
\end{align*}
% DEPOIS EU ARRUMO ISSO
\begin{center}
    \textbf{Previsão de temperatura máxima, em graus Celsius, em São Paulo, diariamente, pelo período de 14 dias, de 07 a 21 de fevereiro de 2025.}
    \begin{tabular}{c | c | c | c | c | c}
        \hline
        i & Temperaturas & $fi$ & $fri$ ( \%) & $Fi$ & $Fri$ (\%) \\
        \hline
        1 & 27           & 1    & 7.14         & 1    & 7.14  \\
        2 & 29           & 1    & 7.14        & 3    & 14.28  \\
        3 & 30           & 3    & 21.42        & 6    & 35.71  \\
        4 & 31           & 7    & 50           & 10   & 85.71  \\
        5 & 32           & 2    & 14.28        & 15   & 100  \\
        \hline
          &              &  $\Sigma fi = 14 $    & $ \Sigma fri = 100 $            &      & 
    \end{tabular}\\
    \textbf{Fonte:} meteoblue --- A Windy.com Company. Disponível em: \url{https://www.meteoblue.com/pt/tempo/14-dias/s\%c3\%a3o-paulo\_brasil\_3448439}. Acesso em 07 fev. 2025.
\end{center}
\soln{b) 3 vezes. 21,42\%.}
\soln{c) 6 vezes. 35.71\%.}
\soln{d) 2 vezes. 14.28\%.}
\newpage
\anyproblem{2- Com os dados brutos abaixo, faça o que se pede a seguir.}
\begin{center}
    \textbf{Idades dos 18 Jogadores de Campo do Brasil das Eliminatória da Copa Libertadores 2023-24}
    \begin{tabular}{llllllllllllllllll}
        \hline 33 & 32 & 27 & 29 & 24 & 22 & 27 & 29 & 27 & 27 & 23 & 28 & 24 & 23 & 23 & 24 & 20 & 17 \\
        \hline
    \end{tabular}
    \textbf{Fonte:} ESPN (Disney +). Disponível em: \url{https://www.espn.com.br/futebol/time/elenco/\_/id/205/liga/FIFA.WORLD}. Acesso em: 07 fev. 2025.
\end{center}
\begin{enumerate}[label=\alph*)]
    \item Faça a \textbf{Tabela de Distribuição de Frequências} (4 tipos de frequências) para \textbf{variável contínua}. \textbf{Use a Regra de Struges}. Apresente a fórmula e os cálculos do \textbf{número de classes $i$} e da \textbf{amplitude das classes $h$}.
    \item Quantos jogadores tem de 17 anos inclusive a 20 anos exclusive? Qual é o percentual correspondente a essa quantidade?
    \item Quantos jogadores tem de 32 anos inclusive a 35 anos exclusive? Qual é o percentual correspondente a essa quantia?
    \item Quantos jogadores tem de 17 anos inclusive a 23 anos exclusive? Qual é o percentual correspondente a essa quantia?
\end{enumerate}

\newpage

\anyproblem{3- Os dados a seguir representam a quantidade de produtos vendidos por 30 vendedores em uma loja, organizados em classes.}
\begin{table}
    \begin{center}
        \begin{tabular}{| c | c | c | c | c | c |}
            \hline
            \textbf{Produtos vendidos} & 1--10 & 11--20 & 21 -- 30 & 31 -- 40 & 41 -- 50 \\
            \hline
            \textbf{Frequência absoluta (fi)} & 3 & 8 & 12 & 4 & 3 \\
            \hline
        \end{tabular}
    \end{center}
\end{table}

\begin{enumerate}[label=\alph*)]
    \item Faça a \textbf{Tabela de Distribuição de Frequências} (4 tipos de frequências). Utilize os mesmos intervalos de classe, aqui representados "---".
    \item A frequência relativa até a classe 31--40 é igual a:
        \begin{enumerate}[label=\textbf{\Alph*)}]
            \item ( ) 13.33\%.
            \item ( ) 75.00\%.
            \item ( ) 80.00\%.
            \item ( ) 85.00\%.
            \item ( ) 90.00\%.
        \end{enumerate}
\end{enumerate}

\newpage
\anyproblem{4- Dado o ROL abaixo, faça o que se pede a seguir.}
% tabela 16 por 2.
\begin{center}
    \textbf{Estaturas, em cm, de 32 alunos do 2º Tecnólogo em Logística, Período Noturno, Fatec-SJC, 2º semestre de 2019}\\
    \begin{tabular}{llllllllllllllll}
        \hline
        150 & 153 & 154 & 160 & 164 & 164 & 165 & 165 & 165 & 166 & 169 & 170 & 170 & 170 & 170 & 170 \\
        170 & 170 & 171 & 172 & 173 & 174 & 175 & 176 & 176 & 177 & 179 & 180 & 183 & 183 & 184 & 190 \\
        \hline
    \end{tabular}
    \textbf{Fonte:} Profª Dra. Nanci de Oliveira
\end{center}
\begin{enumerate}[label=\alph*)]
    \item Faça a \textbf{Tabela de Distribuição de Frequências} com intervalos de classe para \textbf{variável contínua}. Use a \textbf{Regra de Sturges}. Mostre a fórmula e os cálculos do \textbf{número de classes $i$} e da \textbf{amplitude das classes $h$}.
    \item Quantos alunos tem de 164cm inclusive a 171cm exclusive? Qual é o percentual correspondente a essa quantidade?
    \item Quantos alunos tem de 150cm inclusive a 171cm exclusive? Qual é o percentual correspondente a essa quantia?
    \item Quantos alunos tem de 150cm inclusive a 192cm exclusive? Qual é o percentual correspondente a essa quantia?
\end{enumerate}
\end{document}
