\documentclass{jhwhw}
\usepackage{pgfplots}
\usepackage{cancel}
\usepackage{mathtools}
\usepackage{tikz}
\usepackage{amsmath}
\usepackage{multicol}
\usepackage[portuguese]{babel}

\pgfplotsset{compat=1.18}
\usepgfplotslibrary{fillbetween}
\usetikzlibrary{patterns}

\newcommand{\dx}{\mathrm{dx}}
\newcommand{\du}{\mathrm{du}}
\newcommand{\fr}{\mathrm{Fr}}
\begin{document}
\author{Gabriel Vasconcelos Ferreira}
\title{%
	\large Medidas de Distribuição de Frequências \\
	Parte 2 \\
}
\maketitle
\chapter{Medidas de dispersão ou variabilidade}
\section{Variância da amostra}
Fórmula:
\[
	\boxed{S = \sqrt{\frac{\sum{(X_i-\overline{X})^2\cdot f_i}}{n-1}}} \text{ onde: } \begin{cases}
		X_i = \text{variável } \mid X_i = Pmi = \text{pontos médios} \\
		n = \text{total de dados} = \sum{f_i}                        \\
		f_i = \text{frequências das classes}                         \\
		\overline{X} = \text{média aritmética}
	\end{cases}
\]
\section{Desvio padrão da amostra}
Fórmula:
\[
	\boxed{S = \sqrt{\frac{\sum{(X_i-\overline{X})^2\cdot f_i}}{n-1}}} \text{ ou } \boxed{S = \sqrt{\text{Variância}}}
\]
\section{Medida de dispersão relativa}
Coeficiente de variação:
\[
	C.V = \frac{S}{\overline{X}} \cdot 100 (\%)
\]
\chapter{Exemplos}
\section{Exemplo 1}
Os resultados de uma amostra do número de crianças
por família em uma região estão dispostos na tabela
abaixo. Determine a variância e o desvio padrão.
\begin{center}
	\begin{tabular}{c | c | c | c | c }
		\hline
		i & $X_i$ & $f_i$                    & $X_i f_i$                     & $(X_i - \overline{X})^2 f_i$                                    \\
		\hline
		1 & 0     & 10                       & 0                             & $(0-1.8)^2\cdot10 = 32.4$                                       \\
		2 & 1     & 19                       & 19                            & $(1-1.8)^2\cdot10 = 12.16$                                      \\
		3 & 2     & 7                        & 14                            & $(2-1.8)^2\cdot7 = 0.28$                                        \\
		4 & 3     & 7                        & 21                            & $(3-1.8)^2\cdot7 = 10.08$                                       \\
		5 & 4     & 2                        & 8                             & $(4-1.8)^2\cdot2 = 9.68$                                        \\
		6 & 5     & 1                        & 5                             & $(5-1.8)^2\cdot1 = 10.24$                                       \\
		7 & 6     & 4                        & 24                            & $(6-1.8)^2\cdot4 = 70.56$                                       \\
		\hline
		  &       & $\sum^n_{i=1} f_i = 50 $ & $ \sum^n_{i=1} X_i f_i = 91 $ & $\displaystyle\sum_{i=1}^{n}(X_i - \overline{X})^2 f_i = 145,4$
	\end{tabular}\\
\end{center}
Média: \[
	\overline{X} = \frac{\sum_{i=1}^{n} X_i f_i}{\sum_{i=1}^{n} f_i} = \frac{91}{50} = 1.8 \text{ crianças}
\]
Variância: \[
	s^2 = \frac{\sum_{i=1}^{n} (X_i - \overline{X})^2 \cdot f_i}{n-1} = \frac{145.4}{50-1} = 2.967 \text{ crianças}
\]
Desvio padrão: \[
	s = \sqrt{2.967} = 1.7 \text{ crianças}
\]
\section{Exemplo 2}
O resultado de uma sondagem na qual mil adultos foram
indagados sobre quanto gastavam anualmente na preparação
de uma viagem de férias resultou na distribuição de frequência
abaixo. Estime a média, a variância e o desvio padrão amostrais
do conjunto de dados.
\begin{table}
	\centering
	\resizebox{\columnwidth}{!}{%
		\begin{tabular}{c | c | c | c | c | c }
			\hline
			i & Gastos (US\$)    & $f_i$                      & $X_i = Pm_i$                  & $X_i f_i$                                                       & $(X_i - \overline{X})^2 f_i$    \\
			\hline
			1 & 0 $\vdash$ 100   & 380                        & $\frac{(0+100)}{2} = 50 $     & $50 \cdot 380 = 19000$                                          & $(50-189)^2\cdot380 = 7341980$  \\
			\hline
			2 & 100 $\vdash$ 200 & 230                        & $\frac{(100+200)}{2} = 150 $  & $ 150 \cdot 230 = 34500 $                                       & $(150-189)^2\cdot 230 = 349830$ \\
			\hline
			3 & 200 $\vdash$ 300 & 210                        & $\frac{(200+300)}{2} = 250$   & $250 \cdot 210 = 52500$                                         & $(250-189)^2*210 = 781410$      \\
			\hline
			4 & 300 $\vdash$ 400 & 50                         & $\frac{(300+400)}{2} = 350$   & $350 \cdot 50 = 17500$                                          & $(350-189)^2*50 = 1296050$      \\
			\hline
			5 & 400 $\vdash$ 500 & 60                         & $\frac{(400+500)}{2} = 450$   & $450 \cdot 60 = 27000$                                          & $(450-189)^2*60 = 4087260$      \\
			\hline
			6 & 500 $\vdash$ 600 & 70                         & $\frac{(500+600)}{2} = 550$   & $550 \cdot 70 =   38500$                                        & $(550-189)^2*70 = 9122470$      \\
			\hline
              &                  & $\sum^n_{i=1} f_i = 1000 $ &  & $ \sum^n_{i=1} X_i f_i = 189000 $ & $\displaystyle\sum_{i=1}^{n}(X_i - \overline{X})^2 f_i = 22979000$ \\
		\end{tabular}%
	}
\end{table}

Média: \[
	\overline{X} = \frac{\sum_{i=1}^{n} X_i f_i}{\sum_{i=1}^{n} f_i} = \frac{189000}{1000} = 189 \text{ dólares}
\]
Variância: \[
	s^2 = \frac{\sum_{i=1}^{n} (X_i - \overline{X})^2 \cdot f_i}{n-1} = \frac{22979000}{1000-1} = 230002.002 \text{ dólares}
\]
Desvio padrão: \[
	s = \sqrt{230002.002} = 151.66 \text{ dólares}
\]
\end{document}
