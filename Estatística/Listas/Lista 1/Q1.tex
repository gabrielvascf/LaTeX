\soln{Questão 1:}
\soln{a) }
\begin{align*}
    i &= 1 + 3.3 \cdot \log{n}\\
    i &= 1 + 3.3 \cdot \log{14}\\
    i &= 4.7822\dots\\
    i &= \boxed{5}
\end{align*}
Rol $\rightarrow$ 
    \boxed{
        \begin{tabular}{c c c c}
            27 & 30 & 31 & 32 \\
            29 & 31 & 31 & 32 \\
            30 & 31 & 31 &    \\
            30 & 31 & 31 & 
        \end{tabular}
    }

\[
    \fr_i = \frac{fi}{\Sigma fi} 
\]
\begin{align*}
    & \fr_1 = \frac{1}{14} \times 100 = 7.14   & & \fr_4 = \frac{7}{14} \times 100 = 50 \\
    & \fr_2 = \frac{2}{14} \times 100 = 14.28  & & \fr_5 = \frac{2}{14} \times 100 = 14.28 \\
    & \fr_3 = \frac{3}{14} \times 100 = 21.42  & &
\end{align*}
% DEPOIS EU ARRUMO ISSO
\begin{center}
    \textbf{Previsão de temperatura máxima, em graus Celsius, em São Paulo, diariamente, pelo período de 14 dias, de 07 a 21 de fevereiro de 2025.}
    \begin{tabular}{c | c | c | c | c | c}
        \hline
        i & Temperaturas & $fi$ & $fri$ (\%) & $Fi$ & $Fri$ (\%) \\
        \hline
        1 & 27           & 1    & 7.14         & 1    & 7.14  \\
        2 & 29           & 1    & 7.14        & 3    & 14.28  \\
        3 & 30           & 3    & 21.42        & 6    & 35.71  \\
        4 & 31           & 7    & 50           & 10   & 85.71  \\
        5 & 32           & 2    & 14.28        & 15   & 100  \\
        \hline
          &              &  $\Sigma fi = 14 $    & $ \Sigma fri = 100 $            &      & 
    \end{tabular}\\
    \textbf{Fonte:} meteoblue --- A Windy.com Company. Disponível em: \url{https://www.meteoblue.com/pt/tempo/14-dias/s\%c3\%a3o-paulo\_brasil\_3448439}. Acesso em 07 fev. 2025.
\end{center}
\soln{b) 3 vezes. 21,42\%.}
\soln{c) 6 vezes. 35.71\%.}
\soln{d) 2 vezes. 14.28\%.}
