\soln{Questão 4:}
\soln{a) }
Regra de Struges:
\begin{align*}
	i & = 1 + 3.3 \cdot \log{n}  \\
	i & = 1 + 3.3 \cdot \log{32} \\
	i & = 1 + 3.3 \cdot 1.50515  \\
	i & = 1 + 4.966495           \\
	i & = 5.966495               \\
	i & = 6
\end{align*}
\begin{align*}
	\fr_i & = \frac{fi}{\Sigma fi} \\
	\fr_i & = \frac{fi}{32}
\end{align*}
Amplitude das classes:
\begin{align*}
	h & = \frac{AT}{i}      \\
	h & = \frac{190-150}{6} \\
	h & = \frac{40}{6}      \\
	h & = 6.66\dots         \\
	h & = 7
\end{align*}
\begin{align*}
	 & \fr_1 = \frac{3}{32} \times 100 = 9.375  &  & \fr_4 = \frac{8}{32} \times 100 = 25      \\
	 & \fr_2 = \frac{1}{32} \times 100 = 3.125  &  & \fr_5 = \frac{5}{32} \times 100 =  15.625 \\
	 & \fr_3 = \frac{14}{32} \times 100 = 43.75 &  & \fr_6 = \frac{2}{32} \times 100 = 6.25
\end{align*}
% DEPOIS EU ARRUMO ISSO
\newpage
\begin{center}
	\textbf{Estaturas, em cm, de 32 alunos do 2º Tecnólogo em Logística, Período Noturno, Fatec-SJC, 2º semestre de 2019}\\
	\begin{tabular}{c | c | c | c | c | c}
		\hline
		i & Idades           & $fi$              & $fri$ (\%)           & $Fi$ & $Fri$ (\%) \\
		\hline
		1 & 150 $\vdash$ 157 & 3                 & 9.375                & 3    & 9.375      \\
		\hline
		2 & 157 $\vdash$ 164 & 1                 & 3.125                & 4    & 12.5       \\
		\hline
		3 & 164 $\vdash$ 171 & 14                & 43.75                & 18   & 56.25      \\
		\hline
		4 & 171 $\vdash$ 178 & 8                 & 25                   & 26   & 81.25      \\
		\hline
		5 & 178 $\vdash$ 185 & 5                 & 15.625               & 31   & 96.875     \\
		\hline
		6 & 185 $\vdash$ 192 & 1                 & 3.125                & 32   & 100        \\
		\hline
		  &                  & $\Sigma fi = 32 $ & $ \Sigma fri = 100 $ &      &
	\end{tabular}\\
	\textbf{Fonte:} Profª Dra. Nanci de Oliveira
\end{center}
\soln{b) 14 alunos. 43.75\%.}
\soln{c) 18 alunos. 56.25\%.}
\soln{d) 32 alunos. 100\%. (?????)}
