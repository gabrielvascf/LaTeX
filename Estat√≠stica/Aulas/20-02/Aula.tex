\documentclass{jhwhw}
\usepackage{pgfplots}
\usepackage{cancel}
\usepackage{mathtools}
\usepackage{tikz}
\usepackage{amsmath}
\usepackage{multicol}
\usepackage[portuguese]{babel}

\pgfplotsset{compat=1.18}
\usepgfplotslibrary{fillbetween}
\usetikzlibrary{patterns}

\newcommand{\dx}{\mathrm{dx}}
\newcommand{\du}{\mathrm{du}}
\newcommand{\fr}{\mathrm{Fr}}
\begin{document}
\author{Gabriel Vasconcelos Ferreira}
\title{%
    Aula 2 \\ 
    \large Exemplo --- Distribuição de Frequências \\para Variável Contínua (Slide 6)\\
}
\maketitle    
\chapter{Anotações:}\label{chap:Anotações:} % (fold)
\begin{enumerate}
    \item Dados brutos $\rightarrow$ sem ordem
    \item Rol $\rightarrow$ dados em ordem crescente
    \begin{center}
        \boxed{
        \begin{tabular}{c c c c c}
            18 & 20 & 22 & 26 & 34 \\
            18 & 20 & 23 & 28 & 35 \\
            19 & 21 & 25 & 29 & 35 \\
            19 & 22 & 25 & 31 & 
        \end{tabular}
    } $\rightarrow n = 19 \rightarrow$ 
    nº de dados (idades)
    \end{center}
\end{enumerate}

% chapter Anotações: (end)
\section{Regra de Sturges}
nº de classes:
\begin{multline*}
    \boxed{i = 1 + 3.3 \cdot \log{n}}\\
    i = 1 + 3.3 \cdot \log{19}\\
    i = 5.\cancel{2198}\ldots\\
    \boxed{i = 5 \text{ classes (linhas)}}
\end{multline*}
Amplitude das classes:
\[
    h = \frac{AT}{i}
\]
onde \\$AT = \text{Amplitude Total}$\\
$AT = (\text{ maior } - \text{ menor }) \text{ valor do Rol}$
\[
    AT = 35 - 18 = 17 \rightarrow \boxed{AT = 17}
\]
\[
    h = \frac{AT}{i} = \frac{17}{5} = 3.4 \rightarrow \boxed{h = 3}
\]
$h$ é o intervalo entre cada linha. ex: 18 $\vdash$ 21.
\section{Idades dos alunos do 3º Banco de Dados, Fatec SJC, 22/08/19}
\begin{center}
    \begin{tabular}{c | c | c | c | c | c}
        \hline
        i & Idades & $fi$ & $fri$ ( \%) & $Fi$ & $Fri$ (\%) \\
        \hline
        1 & 18 $\vdash$ 21 & 6 & 31,58 & 6 & 52,63k  \\
        2 & 21 $\vdash$ 24 & 4 & 21,05 & 10 & 52,63k \\
        3 & 24 $\vdash$ 27 & 3 & 15,79 & 13 & 52,63k \\
        4 & 27 $\vdash$ 30 & 2 & 10,53 & 15 & 52,63k \\
        5 & 30 $\vdash$ 33 & 1 & 5,26 & 16 & 52,63k \\
        6 & 33 $\vdash$ 36 & 3 & 15,79 & 19 & 52,63k \\
        \hline
          &                & $\Sigma fi = 19$  & $ \Sigma fi = 100 $      &    & 
    \end{tabular}
\end{center}
\subsection{Cálculos relevantes:}
\[
    \boxed{\fr_i = \frac{fi}{\Sigma fi}}
\]
\[
    \fr_1 = \frac{Fi}{\Sigma fi} = \frac{6}{19} \times 100 = \cancelto{\boxed{31.58}}{31.578}
\]
\[
    \fr_2 = \frac{F_2}{\Sigma fi} = \frac{10}{19} \times 100 = \boxed{52.63\cancel{1}}
\]
\[
    \fr_3 = \frac{F_3}{\Sigma fi} = \frac{13}{19} \times 100 = \boxed{68.42\cancel{1}}
\]
\[
    \fr_4 = \frac{F_3}{\Sigma fi} = \frac{13}{19} \times 100 = \boxed{68.42\cancel{1}}
\]
\[
    \fr_5 = \frac{F_3}{\Sigma fi} = \frac{13}{19} \times 100 = \boxed{68.42\cancel{1}}
\]
\[
    \fr_6 = \frac{F_3}{\Sigma fi} = \frac{13}{19} \times 100 = \boxed{68.42\cancel{1}}
\]
\end{document}

