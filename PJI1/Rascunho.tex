\documentclass{article}

\begin{document}
\section{Metodologia}
Ao considerar as dificuldades enfrentadas por alunos com desigualdades sociais, decidimos centralizar o projeto no desenvolvimento de um recurso que permita que pessoas que não sejam alfabetizadas ou não utilizem o Sistema Braille, possam ter autonomia em sua orientação e mobilidade nos espaços públicos.
O projeto será focado em um objeto que, quando aproximado de etiquetas \emph{RFID}, fará uso de um microcontrolador para interpretar dados recebidos pela etiqueta para auxiliar o usuário a se localizar pelo espaço utilizando pontos de interesse através de instruções por voz.
O microcontrolador será alimentado por um carregador portátil dentro da estrutura do objeto, que também poderá reproduzir sons quando necessário através de um módulo adicional.
\section{Resultados esperados}
Através do desenvolvimento do projeto, buscamos a criação de uma ferramenta que permita que o usuário identifique salas de aula e pontos de interesse de maneira rápida, replicável e simples.
\end{document}